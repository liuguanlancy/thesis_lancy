% ==========================================================================================
% PREAMBLE
% ==========================================================================================
%----------------------------------------------------------------------------------------
% GENERAL  - PACKAGES
%----------------------------------------------------------------------------------------
% GENERAL
\usepackage{etex} %Because of many packages --> Extended TeX.
\usepackage[utf8]{inputenc} %Due to vowels.
% \usepackage[british]{babel} %Define the language style.

%Load some mathematical packages.
\usepackage{amsmath}
\usepackage{amsfonts}
\usepackage{amsmath}
\usepackage{amssymb}
\usepackage{mathtools}
\usepackage{breqn} % breaking of equations (be careful with using ENDFLOAT  with this)

%  LAYOUT/PAGE/TEXT FORMATTING
\usepackage[left=1in, right=1in]{geometry} %Helps to structure the paper layout.
\usepackage{setspace} %Use double spacing.

\usepackage[Lenny]{styles/fncychap} %Design of the thesis. = Fancy chapter

\usepackage{fancyhdr} %To customize the headers and footers.
\usepackage[hang,bottom,stable,multiple]{footmisc} %Style of footnotes.
\usepackage{dsfont} %Nice style for the indicator function.

\usepackage[svgnames]{xcolor} % Enabling mixing colors and color's call by 'svgnames'
% define new colors (not limited to text obviously)
\definecolor{MyColor1}{rgb}{0.2,0.4,0.6} % mix personal color


% FLOATS
\usepackage{booktabs} %In case you need \cmidrule or \addlinespace in tables.
\usepackage{array} %To create tables and matrices.
\usepackage{hhline}
\usepackage{rotating} %To rotate a table/figure. e.g. \sidewaystable
\usepackage{tabularx} %An extended version of tabular.
\usepackage{float} %Allows for the 'H' option

\usepackage{graphicx} %For the graphics


\usepackage[margin=10pt, font=small, labelfont=bf, labelsep=endash]{caption} %Customize the captions.


% OTHER ENVIRONMENTS


\usepackage{textcomp}

\usepackage{amsthm} %For theorems, definitions etc.
\usepackage{thmtools} %For theorems, definitions etc.

\usepackage{appendix} %For the \appendixpage command.
\usepackage{etoolbox} %To remove the page number on \appendixpage.
\makeatletter %Remove page number on \appendixpage.
\patchcmd{\@chap@pppage}{\thispagestyle{plain}}{\thispagestyle{empty}}{}{}
\makeatother


\usepackage{styles/mcode} %To implement Matlab code.
\usepackage{listings} % For including code in your pdf

% VARIA  - does not mean useless
\usepackage{epstopdf} %For inserting .eps files into the document.
\usepackage{lipsum} %For the \lipsum command to generate a text.
\usepackage{datetime} %For the specification of the date.
\usepackage{chngcntr} %To use counterwithout.
\usepackage{xparse} %Load for \NewDocumentCommand command.
\usepackage{arydshln} %Due to the capability to draw horizontal/vertical dash-lines.

%  REFERENCING
% % BIBLIOGRAPHY - BIBTEX
% \usepackage[sort,round]{natbib} %For the bibliography.
% \bibliographystyle{abbrvnat} %Reference style.

% BIBLIOGRAPHY - BIBLATEX
\usepackage[
backend=biber,
style=apa,
bibstyle=authoryear,
citestyle=authoryear,
maxcitenames=2,
maxbibnames=99
]{biblatex}

\setlength\bibitemsep{1\itemsep} % spacing between entries in references
\addbibresource{references.bib} % .bib file


\usepackage{hyperref} %Must be loaded at the end.
\hypersetup{ %Setup of the reference links.
     colorlinks=false,
     linkcolor=blue,
     citecolor=blue,
     filecolor=magenta,
     urlcolor=blue
}

\usepackage[nameinlink,capitalize]{cleveref} %For the command \cref, load after hyperref.

%----------------------------------------------------------------------------------------
% GENERAL  - SETUP
%----------------------------------------------------------------------------------------
%Define some reasonable margins.
\setlength{\textwidth}{6.6in}
\setlength{\textheight}{8.8in}
\setlength{\topmargin}{-0.1in}
\setlength{\oddsidemargin}{0in}
\setlength{\parskip}{1mm}

\setlength{\parindent}{0cm} %Uncomment this if you don't want to have indents.

%Read just the numbering.
\counterwithout{footnote}{chapter}
\numberwithin{equation}{chapter}

\allowdisplaybreaks[1] %Page breaks of equations are allowed, but avoided if possible. 2-4 more relaxed.

%----------------------------------------------------------------------------------------
% CUSTOM COMMANDS/ENVIRONMENTS
%----------------------------------------------------------------------------------------
%New command for the differential d to have an ordinary d.
\makeatletter
  \newcommand{\ud}{\mathrm{d}}
\makeatother

%Declare Definitions, Theorems etc.(ENVIRONMENTS)
\declaretheorem[style=definition,qed=\(\blacktriangleleft\), numberwithin=chapter]{remark} %additional options; numberwithin=,..., see 'Thmtools' Users’ Guide
\declaretheorem[style=definition,qed=\(\triangle\),numberwithin=chapter]{definition}
\newtheorem{ass}{Assumption}[chapter]
\newtheorem{prop}{Proposition}[chapter]
\newtheorem{lemma}{Lemma}[chapter]
\declaretheorem[style=definition,qed=\(\perp\),numberwithin=chapter]{example}
\newtheorem{theorem}{Theorem}[chapter]
\newtheorem{coroll}{Corollary}[chapter]

%----------------------------------------------------------------------------------------
% TITLE PAGE -  Creates titlepage command
%----------------------------------------------------------------------------------------
%New command for the UZH logo. Used within big \titleGP
\newcommand*{\uzhlogo}{\includegraphics{Graphics/uzh_logo_e_pos.pdf}}
\newcommand*{\ethlogo}{\includegraphics{Graphics/ETHlogo.pdf}}


\newcommand*{\titleGP}{\begingroup %Create the command for including the title page in the document.

\centering %Center all text.

% Include University logo
\vspace*{\baselineskip} %White space at the top of the page.
\ifQF
\uzhlogo\hspace{120pt}\ethlogo\\[2\baselineskip] %University Logo.
\else
\uzhlogo\\[2\baselineskip] %University Logo.
\fi

% Create title 'box'
\rule{\textwidth}{1.6pt}\vspace*{-\baselineskip}\vspace*{2pt} %Thick horizontal line.
\rule{\textwidth}{0.4pt}\\[\baselineskip] %Thin horizontal line.

% TITLE
{\LARGE [ Title ]}\\[0.2\baselineskip] %Title.

\rule{\textwidth}{0.4pt}\vspace*{-\baselineskip}\vspace{3.2pt} %Thin horizontal line.
\rule{\textwidth}{1.6pt}\\[2\baselineskip] %Thick horizontal line.
\scshape %Small caps.

% Thesis
\thesis's Thesis\\[2\baselineskip]

\ifQF
Submitted in partial fulfillment of the requirements for the degree of Master of Science in Quantitative Finance \par
\else
Submitted in partial fulfillment of the requirements for the degree of \thesis\ of Arts in Economics and Business Administration \par
\fi

\vspace*{2\baselineskip}

% AUTHOR BLOCK - ADAPT
Author\\
{\Large [ Your Name ] \\ [5pt]
 }
% [ Address ] \\[5pt]
[ Student-ID ]\\[5pt]
[ Contact (E-Mail) ] \\

\vspace*{2\baselineskip}

% SUPERVISOR+ASSISTANT BLOCK -ADAPT
Supervisor\\
{\Large Prof.\ Dr.\ Markus Leippold\\[5pt]
\small Professor of Financial Engineering\\[5pt]
\small Department of Finance\\[5pt]University of Zurich\par}
\vspace*{2\baselineskip}

Assistant\\

{\Large [ Name ] \par}

\vfill

{\scshape Date of Submission: [ Date ]} \\[0.3\baselineskip]

\endgroup} % This is the end of the command


%----------------------------------------------------------------------------------------
% HEADER/FOOTER
%----------------------------------------------------------------------------------------
% Special header and footer style for the Executive summary and Task Assignment section.
\fancypagestyle{firststyle}{%
  \fancyhf{}%
  \renewcommand{\headrulewidth}{0pt}
  \fancyfoot[C]{\thepage}
}

%Customize headers and footers. - ADAPT
\pagestyle{fancy}
\fancyhead[R]{\thepage}
\fancyhead[L]{\rightmark}
\fancyfoot[L]{[ Your Name ]} % NAME
\fancyfoot[C]{}
\fancyfoot[R]{[ Running Thesis Title ]} % RUNNING TITLE
\setlength{\headheight}{13.6pt}

%----------------------------------------------------------------------------------------
% SIGNATURE setup
%----------------------------------------------------------------------------------------
%Define the signature line with dots. (create \dotbox command)
\NewDocumentCommand\dotbox{o O{.5\linewidth} m O{3ex} O{\linewidth}}
{
  \begin{minipage}{7cm}
    \makebox[7cm][l]{\,.\dotfill}
    \\
    \makebox[7cm][l]{\,#3}
  \end{minipage}
}
