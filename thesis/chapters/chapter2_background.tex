\chapter{Background and Related Work}

This chapter reviews the areas of research that inform our study of data mixture effects in financial language model pretraining. We start with an overview of financial NLP, then discuss pretraining fundamentals, examine prior work on mixture strategies, and close with domain adaptation and transfer learning. Put another way, we move from context to mechanisms to practice.

\section{Financial NLP}

\subsection{The Financial NLP Landscape}

Financial natural language processing covers many tasks: sentiment on news and social media, question answering on regulatory documents, numerical reasoning in reports, and information extraction from SEC filings \parencite{araci2019finbert, chen2021finqa}. The domain has some specific challenges compared to general-domain NLP: specialized vocabulary (e.g., ``alpha'', ``beta'', ``EBITDA''), domain reasoning patterns (e.g., causal chains in market analysis), numerical grounding (financial statements), and temporal dynamics (market events, earnings releases) \parencite{wu2023bloomberggpt, araci2019finbert}.

\subsection{Existing Financial Language Models}

Several finance-focused language models have appeared in recent years. \textbf{BloombergGPT} \parencite{wu2023bloomberggpt}, a 50-billion-parameter model, was pretrained on a mixture of 51\% financial and 49\% general data, showing strong performance on financial benchmarks while keeping general capabilities. \textbf{FinBERT} variants \parencite{araci2019finbert, yang2020finbert} adapted BERT to financial text via continued pretraining, improving sentiment analysis on financial news. More recently, \textbf{FinGPT} \parencite{yang2023fingpt} explored open-source instruction-tuning for financial tasks.

\subsection{Domain-Specific Challenges}

Financial NLP faces three critical challenges. \textbf{First}, privacy concerns: financial institutions cannot upload sensitive data (portfolios, trading strategies, client information) to external APIs, so locally deployable models are needed \parencite{wu2023bloomberggpt}. \textbf{Second}, data scarcity: compared to general web text, curated financial corpora are smaller, so data-efficient training is important. \textbf{Third}, rapid vocabulary change: financial language shifts with market trends (e.g., ``DeFi'', ``ESG''), so models must adapt to new terms.

\section{Language Model Pretraining}

\subsection{Pretraining Objectives and Architecture}

Modern language models are mostly trained using the \textbf{causal language modeling} objective: predict the next token given the context \parencite{radford2019language, brown2020language}. This self-supervised approach learns from large unlabeled corpora. Architecturally, transformer-based decoder-only models (GPT family, LLaMA, Qwen) are now the standard design, with multi-head self-attention capturing long-range dependencies and feed-forward layers providing non-linear transformations \parencite{vaswani2017attention, touvron2023llama}.

\subsection{Scaling Laws and Model Size Effects}

The seminal work of \textcite{kaplan2020scaling} established power-law relationships between model size, dataset size, compute budget, and final performance. Their key finding—that larger models are more sample efficient—motivated the move toward billion-parameter models. Later work added nuance: \textcite{hoffmann2022training} showed models are often undertrained relative to size (Chinchilla scaling), and \textcite{tay2022ul2} showed that objectives and data quality shape scaling behavior.

Critically, \textbf{hyperparameter sensitivity} has received less attention. While \textcite{mccandlish2018empirical} noted that optimal learning rates can decrease with model size, systematic studies for models in the 0.6B–4B range—especially in specialized domains—remain limited. Many scaling-law papers assume proper tuning without detailing the process, which can hide training dynamics. In our work, all main runs used LR=2e-5; in a few cases we reduced LR pragmatically to stabilize training. We do not claim a general learning-rate rule.

\subsection{Computational and Memory Considerations}

Training large language models requires substantial computational resources. A 1-billion-parameter model with 32-bit precision consumes approximately 4GB of memory for parameters alone, with optimizer states (e.g., Adam's momentum terms) doubling or tripling this requirement \parencite{rajbhandari2020zero,kingma2014adam}. For models in the 0.6B–4B range targeted in this thesis, memory-efficient techniques like mixed-precision training (bfloat16), gradient accumulation, activation checkpointing, and parameter-efficient fine-tuning methods such as LoRA enable training on enterprise-class GPUs (e.g., NVIDIA RTX A6000 48GB, A100 40GB, H100 80GB) \parencite{narayanan2021efficient,hu2021lora}.

\section{Data Mixture Strategies}

\subsection{Curriculum Learning and Sequential Mixing}

\textbf{Curriculum learning} in language model pretraining involves carefully sequencing training data from easier to harder examples, or from general to specialized domains \parencite{bengio2009curriculum}. \textcite{wu2022opt} applied curriculum strategies in pretraining OPT models, progressively increasing data difficulty. In the financial domain, a natural curriculum might proceed from general Wikipedia text to financial news to technical SEC filings. However, empirical evidence for curriculum's effectiveness in large-scale pretraining remains mixed across objectives and domains \parencite{longpre2023pretrainer}. Some works report limited gains for masked language modeling at scale, while others show improvements in specialized settings; in practice, many production systems rely on mixture-based sampling rather than strict curricula \parencite{raffel2020exploring,wu2022opt}.

\subsection{Simultaneous Mixture Approaches}

An alternative to sequential mixing is \textbf{simultaneous mixture}: sampling from multiple datasets concurrently throughout training. \textcite{raffel2020exploring} (T5) used a multi-task mixture with task-specific prefixes, finding that diverse pretraining improved downstream task generalization. \textcite{xie2023doremi} introduced DoReMi, a method that dynamically adjusts domain mixture weights during training based on validation perplexity, achieving better sample efficiency than static mixtures on The Pile dataset.

\textbf{BloombergGPT's approach} \parencite{wu2023bloomberggpt} is particularly relevant: they mixed 51\% financial data with 49\% general-purpose data (The Pile, C4) at the token level, demonstrating that balanced mixtures preserve general capabilities while gaining domain expertise. However, their work focused on a single 50B model; the interaction between mixture strategy and model size (0.6B vs 4B) remains underexplored. Our work tests this hypothesis systematically across three model scales, finding that mixed financial datasets (21.55 ppl @ 4B) substantially outperform both Wiki+Financial mixtures (26.69 ppl @ 4B, 24\% degradation) and pure WikiText (41.96 ppl mean financial @ 4B, 95\% degradation), as documented in \Cref{fig:scaling_comparison_all,tab:mixed_financial_results,tab:mixed_wiki_financial_results}. This suggests that domain purity may be more valuable than domain balance for specialized applications.

\subsection{Domain Proportions and Sampling Strategies}

Determining optimal domain proportions in mixtures is non-trivial. Three sampling strategies dominate the literature:

\textbf{1. Temperature sampling} \parencite{arivazhagan2019massively}: Sample from dataset $d$ with probability $p_d \propto n_d^{1/T}$ where $n_d$ is dataset size and $T$ is temperature. $T < 1$ upsamples small datasets; $T > 1$ downsamples them.

\textbf{2. Capping strategies} \parencite{longpre2023pretrainer}: Cap the largest dataset(s) at a threshold (e.g., 50\% of total tokens) to prevent dominance, then proportionally sample others. This ensures diversity even when one dataset is orders of magnitude larger.

\textbf{3. Equal mixing} \parencite{sanh2022multitask}: Assign equal sampling probability to each dataset regardless of size. This maximizes task diversity but may undersample large datasets.

This thesis employs a \textbf{50\% capping strategy} (``50cap'') for financial dataset mixtures, as described in Chapter 3, to balance diversity with data efficiency. We chose it for simplicity and stability in our setup.

\section{Domain Adaptation and Transfer Learning}

\subsection{Cross-Domain Transfer in Language Models}

\textbf{Transfer learning}—pretraining on broad data then fine-tuning on specialized tasks—has been the common approach since BERT \parencite{devlin2019bert,pan2010transfer,zhuang2020comprehensive}. The assumption is that general linguistic knowledge transfers to domain applications. However, recent work shows nuance: \textcite{gururangan2020don} found that \textbf{domain-adaptive pretraining} (continued pretraining on domain corpora) improves performance across domains, suggesting general pretraining alone is not enough for specialized use.

In finance, \textcite{araci2019finbert} showed improvements from continued pretraining on financial news; \textcite{yang2020finbert} saw further gains with task-adaptive pretraining. More recently, \textcite{huang2023finbert} found that domain-specific pretraining outperforms general models on financial information extraction. However, these studies focus on BERT-style masked language models and classification tasks—the effectiveness of domain adaptation for \textit{generative causal language models} in financial pretraining is less studied. Advances in parameter-efficient fine-tuning, such as surgical fine-tuning \parencite{lee2022surgical}, suggest selective adaptation may improve transfer while mitigating catastrophic forgetting.

\subsection{Catastrophic Forgetting and Stability}

A key challenge in domain adaptation is \textbf{catastrophic forgetting}: when a pretrained model is further trained on domain-specific data, it may lose general knowledge \parencite{mccloskey1989catastrophic, french1999catastrophic}. \textcite{kirkpatrick2017overcoming} introduced Elastic Weight Consolidation (EWC) to mitigate forgetting by penalizing changes to important parameters. In the context of data mixtures, \textit{simultaneous mixing} of general and domain data can act as a form of implicit regularization, reducing forgetting by continuously exposing the model to diverse distributions \parencite{arivazhagan2019massively,raffel2020exploring}.

\subsection{Distribution Shift and Domain Mismatch}

\textbf{Distribution shift}—the discrepancy between training and evaluation data—directly impacts generalization \parencite{quinonero2009dataset}. In financial NLP, this appears as vocabulary shift (financial terminology vs general language), discourse differences (analytical reports vs encyclopedic text), and formatting (structured tables in 10-K filings vs narrative news). \textcite{aharoni2020unsupervised} showed that domain mismatch can severely degrade performance on out-of-distribution test sets, motivating diverse mixtures that cover multiple sub-domains.

Our thesis investigates this empirically: does pretraining purely on high-quality general corpora (WikiText) transfer to financial evaluation sets? Or does domain mismatch make in-domain pretraining necessary? And when mixing in-domain datasets (sentiment, Q\&A, news, reports), do models generalize better than single-dataset training?

\subsection{Related Empirical Studies}

Several empirical studies inform our methodology. \textcite{xie2023doremi} demonstrated that dynamic mixture optimization can outperform static mixtures on The Pile, but their approach requires validation data and multiple training runs, limiting practicality. \textcite{longpre2023pretrainer} surveyed practitioners' mixture strategies, finding that capping strategies and temperature sampling are most common in production settings. \textcite{mitra2023orca2} (Orca-2) showed that training on diverse instruction formats improves reasoning generalization, suggesting that \textit{intra-domain diversity} (multiple financial datasets) may be as important as domain specialization.

Notably absent from prior work are systematic studies of \textbf{dataset size effects} on mixture strategies: when is a dataset large enough for standalone pretraining? When does mixing help vs hurt? And how do these patterns interact with model size? These questions motivate our experimental design in Chapter 3.
