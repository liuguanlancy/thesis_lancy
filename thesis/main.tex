\documentclass[11pt,a4paper,english,oneside]{book}

%----------------------------------------------------------------------------------------
% THESIS SETTINGS - ADAPT
%----------------------------------------------------------------------------------------
\newif\ifQF % default behaviour is false, so NOT in QF
% \QFtrue % Uncomment if you are in the QF program

\newcommand{\thesis}{Master}

% Preamble for shortened thesis
\usepackage{etex}
\usepackage[utf8]{inputenc}

% Math
\usepackage{amsmath}
\usepackage{amsfonts}
\usepackage{amssymb}
\usepackage{mathtools}
\usepackage{breqn}

% Layout
\usepackage[left=1in, right=1in]{geometry}
\usepackage{setspace}
\usepackage{fancyhdr}
\usepackage[hang,bottom,stable,multiple]{footmisc}
\usepackage{dsfont}
\usepackage[svgnames]{xcolor}

% Floats
\usepackage{booktabs}
\usepackage{multirow}
\usepackage{colortbl}
\usepackage{array}
\usepackage{hhline}
\usepackage{rotating}
\usepackage{tabularx}
\usepackage{float}
\usepackage{graphicx}
\usepackage[margin=10pt, font=small, labelfont=bf, labelsep=endash]{caption}

% Other
\usepackage{textcomp}
\usepackage{amsthm}
\usepackage{thmtools}
\usepackage{appendix}
\usepackage{etoolbox}
\usepackage{listings}
\usepackage{epstopdf}
\usepackage{datetime}
\usepackage{chngcntr}
\usepackage{xparse}
\usepackage{arydshln}

% Bibliography
\usepackage[
  backend=biber,
  style=apa,
  bibstyle=authoryear,
  citestyle=authoryear,
  maxcitenames=2,
  maxbibnames=99
]{biblatex}
\addbibresource{../thesis/references.bib}

% Hyperref and cleveref
\usepackage{hyperref}
\hypersetup{
  colorlinks=false,
  linkcolor=blue,
  citecolor=blue,
  filecolor=magenta,
  urlcolor=blue
}
\usepackage[nameinlink,capitalize]{cleveref}

% Setup
\setlength{\textwidth}{6.6in}
\setlength{\textheight}{8.8in}
\setlength{\topmargin}{-0.1in}
\setlength{\oddsidemargin}{0in}
\setlength{\parskip}{1mm}
\setlength{\parindent}{0cm}
\counterwithout{footnote}{chapter}
\numberwithin{equation}{chapter}
\allowdisplaybreaks[1]

% Title helpers
\newcommand*{\titleShort}{\begingroup
\centering
\vspace*{\baselineskip}
{\LARGE Understanding Data Mixture Effects in Financial Language Model Pretraining\\[0.5\baselineskip]
Short Version}\\[1.2\baselineskip]
{\large Guanlan Liu}\\[0.5\baselineskip]
{\small Department of Finance, University of Zurich}\\[0.5\baselineskip]
{\small Date: \today}\\[2\baselineskip]
\endgroup}

 % Main preamble file

\begin{document}

%----------------------------------------------------------------------------------------
% TITLE PAGE
%----------------------------------------------------------------------------------------
\thispagestyle{empty}
\titleGP

\newpage

% \doublespacing
\setcounter{page}{1}
\pagenumbering{Roman}

%----------------------------------------------------------------------------------------
% ABSTRACT
%----------------------------------------------------------------------------------------
\section*{Abstract}
\thispagestyle{firststyle}

We present a compute-normalized study of pretraining data composition for financial language models. We adapt the decoder-only Qwen3 Base architecture to a fixed 100M-token budget with heterogeneous financial texts and general texts with a unified eight-dataset evaluation. We further develop systematic comparisons of individual datasets versus mixtures to evaluate optimal pretraining strategies. In particular, we find that \textbf{medium individual datasets (3.6–8.5M tokens) consistently outperform mixtures on both performance and consistency}. FiQA (6.80 ppl, 19\% spread), FinGPT (7.03 ppl, 37\% spread), and Alpaca (8.73 ppl, 11.5\% spread) achieve 2.5–3.2$\times$ better perplexity AND 1.5–4.8$\times$ better cross-dataset consistency than our seven-source financial mixture (21.55 ppl, 55\% spread). This finding challenges conventional wisdom that data diversity improves robustness. This occurs through a three-way interaction: medium datasets achieve optimal epoch counts (12–28 epochs) with format consistency, while large datasets undertrain (<1 epoch) and large mixtures add format conflicts that small models (0.6B–4B) cannot reconcile. Small datasets (<1M tokens) overtrain (143–352 epochs), leading to memorization. WikiText shows competitive performance at small scales (0.6B: 9.68 ppl) but reverse scaling at larger sizes due to training instability.

Our contributions in this work are three-folded:

\begin{itemize}
    \item[a.] We systematically compare individual datasets versus mixtures via token-matched training and unified eight-dataset evaluation, revealing that medium individual datasets (3.6–8.5M tokens) consistently outperform mixtures on both performance and consistency metrics.

    \item[b.] To understand why mixtures fail, we analyze format inconsistency, vocabulary dilution, and multi-task interference effects. We find that focused optimization on single datasets beats diverse mixing—format consistency and concentrated vocabulary exposure outweigh anticipated diversity benefits.

    \item[c.] We establish that data quality and focus matter more than scale: medium datasets (FiQA 3.6M, FinGPT 4.1M, Alpaca 8.5M, SEC 8.1M) substantially outperform large datasets (News 194M, WikiText 124M). This non-monotonic size-performance relationship reflects optimal epoch counts (12–28) combined with format consistency, outperforming both undertrained large datasets (<1 epoch) and overtrained small datasets (143–352 epochs).
\end{itemize}

\newpage

%----------------------------------------------------------------------------------------
% TABLE OF CONTENTS
%----------------------------------------------------------------------------------------
\tableofcontents
\listoffigures
\listoftables

\newpage
\pagenumbering{arabic}

%----------------------------------------------------------------------------------------
% MAIN CHAPTERS
%----------------------------------------------------------------------------------------

\chapter{Introduction}

% Largely keep the original introduction for context and motivation
% Reuse the existing text by referencing the original chapter content
\chapter{Introduction}

\section{Motivation}

Large language models (LLMs) have changed how we do natural language processing \parencite{vaswani2017attention,radford2019language,brown2020language,touvron2023llama}. Fast. But using them in finance still brings practical hurdles. Financial institutions and individuals handle highly sensitive data—transactions, portfolios, trading strategies—that cannot be sent to external APIs for privacy and competitive reasons (e.g., GDPR) \parencite{eu2016gdpr}. So we need lightweight, locally runnable financial language models that keep reasonable performance while protecting data. No exceptions there.

In practice, domain adaptation tends to follow two paths: train very large models from scratch or fine-tune general models on domain data. Most teams cannot afford the first; the second often misses domain nuances \parencite{gururangan2020don}. And there is a related belief: high‑quality general corpora (e.g., Wikipedia, The Pile) always help specialized applications. Not always; evidence is thinner than many assume \parencite{gao2020pile,raffel2020exploring,longpre2023pretrainer}.

This thesis studies how different data sources—both in‑domain financial data and out‑of‑domain high‑quality corpora—interact during pretraining. We focus on models in the 0.6B to 4B parameter range, which are realistic for laptops and some mobile devices while keeping acceptable performance \parencite{yang2024qwen2,xia2023sheared,team2024gemma,javaheripi2023phi}. Through systematic experiments across 10 pretraining configurations and three model sizes, we present evidence about data mixture strategies for specialized domains \parencite{wu2023bloomberggpt}. Put another way, we test simple choices that many teams can replicate.

This study is timely. Regulations such as GDPR and emerging financial data protection standards push for on‑device processing \parencite{eu2016gdpr}. And as more teams adopt AI with limited compute, insights on 0.6B–4B models are practically useful. Still, we keep scope modest.

Beyond applications, we also add to understanding how models learn from different data distributions. We observed a surprising pattern sometimes called ``reverse scaling''—in some regimes smaller models outperform larger ones. In our runs, these cases point to hyperparameter choices rather than fundamental limits \parencite{kaplan2020scaling,hoffmann2022training,mccandlish2018empirical}. Tuning matters.

\section{Research Questions}

This thesis investigates the following core research questions. We keep them narrow and testable.

\textbf{RQ1: Data Mixture Composition}
How do combinations of in‑domain financial datasets and out‑of‑domain general corpora affect model performance and generalization? Specifically, does mixing multiple financial datasets improve consistency compared to single‑dataset training, and does adding high‑quality general text (WikiText) help or hurt financial tasks? Our results (\Cref{fig:scaling_comparison_all,tab:mixed_financial_results,tab:mixed_wiki_financial_results}) indicate that mixed financial datasets achieve 21.55 ppl (mean across financial evaluations), compared to 26.69 ppl for Wiki+Financial mixtures (overall mean across eight evaluations) and 41.96 ppl for pure WikiText (mean across financial evaluations after LR adjustment)—suggesting in‑domain diversity is the better choice.

\textbf{RQ2: Model Size and Training Dynamics}
How do optimal training configurations vary across model sizes (0.6B, 1.7B, 4B parameters)? What is the relationship between size and learning rate sensitivity? In our setup, we trained all main runs with LR=2e‑5; for a few abnormal cases, we reduced LR pragmatically (e.g., to $1\times10^{-5}$ or $5\times10^{-6}$) and saw improved stability. We do not claim a general scaling rule.

\textbf{RQ3: Dataset Size Effects}
What is the minimum dataset size for effective standalone pretraining, and how does size affect overtraining and cross‑dataset generalization? At what point do small datasets need mixing? With our data, datasets $>$100M tokens enable stable training (\Cref{fig:scaling_news_articles,fig:scaling_sec_reports}), while datasets $<$20M tokens require mixing due to extreme overtraining and high cross‑dataset variability (\Cref{fig:scaling_financial_qa,fig:scaling_twitter,tab:cross_financial_qa,tab:cross_twitter}).

\textbf{RQ4: Domain Transfer Patterns}
How well do models pretrained on financial data transfer across task types (sentiment, question answering, document understanding), and how much does document format matter? Cross‑dataset comparison tables (\Cref{tab:cross_financial_news,tab:cross_financial_repor,tab:cross_alpaca,tab:cross_fingpt,tab:cross_fiqa,tab:cross_twitter}) suggest that format consistency (long‑form, instruction, short‑form) drives transfer more than domain vocabulary, with boldface patterns clustering along format‑based diagonals.

These questions are addressed through a detailed experimental framework with 30 trained models and 237 evaluation results across eight held-out test sets (Mixed Financial excludes WikiText evaluation), providing systematic evidence on data mixture effects in specialized-domain pretraining.

\section{Contributions}

This thesis makes six primary contributions to understanding data mixture effects and training dynamics for language model pretraining:

\textbf{1. Empirical Data Mixture Guidelines}
We provide concrete, evidence-based recommendations for financial language model pretraining, showing that in-domain diversity outperforms high-quality general corpora for specialized domains. Our experiments show that mixed financial datasets achieve 21.55 perplexity at 4B parameters compared to 41.96 perplexity (mean across financial evaluations after LR adjustment) for WikiText pretraining—a 1.95$\times$ gap. These findings challenge the assumption that general high-quality text always helps domain adaptation. We support this with 11 scaling figures and 18 tables (10 per-training-dataset and 8 cross-dataset comparisons).

\textbf{2. Learning Rate Notes}
All main experiments used LR=2e-5. In three follow-ups with abnormalities (WikiText, Financial QA, Twitter), we reduced LR (e.g., to $1\times10^{-5}$ or $5\times10^{-6}$) and observed improved stability and results. We present these as pragmatic fixes in our setup, not a general rule. Figures \Cref{fig:scaling_wikitext,fig:scaling_financial_qa,fig:scaling_twitter} show recovery in these runs; \Cref{tab:financial_qa_lr_comparison,tab:twitter_lr_comparison} give the metrics.

\textbf{3. Dataset Size Effects on Pretraining}
We summarize empirical relationships between dataset size and training viability:
\begin{itemize}
    \item Small datasets (< 20K samples): Extreme overtraining (67-249 epochs), high variance (70-97\% relative spread), require mixing
    \item Medium datasets (20-100K samples): Moderate overtraining (6-30 epochs), acceptable for specific use cases
    \item Large datasets (> 100K samples): Minimal overtraining (2-24 epochs), viable for standalone pretraining
\end{itemize}
These findings provide practical guidance on when mixing is necessary versus when individual datasets suffice, with direct implications for practitioners planning limited data collection and annotation budgets.

\textbf{4. Cross-Domain Interaction Analysis}
We study how high-quality general corpora (WikiText) interact with domain-specific financial data during pretraining. Contrary to common expectations, WikiText provides little benefit and sometimes degrades financial task performance. Mixed WikiText+Financial pretraining achieves 26.69 perplexity compared to 21.55 for pure financial mixing—a 24\% degradation. This suggests domain-specific data strategies are better for specialized applications. Cross-dataset tables show this pattern: WikiText rows rarely capture best-performance positions across financial evaluation columns, while mixed financial rows often do.

\textbf{5. Lightweight Financial Model Feasibility}
We show that 0.6B–4B parameter models can achieve practical financial NLP performance with appropriate mixtures and tuning, enabling privacy-preserving edge deployment. Our 4B model achieves 21.55 perplexity on diverse financial tasks, competitive with much larger models while remaining deployable on consumer hardware. This addresses the need for locally runnable financial AI systems.

\textbf{6. Open-Source Training Pipeline}
We provide a reproducible codebase for mixture-based pretraining with a detailed evaluation framework across 10 experiments and 30 trained models. The pipeline supports automatic mixture composition, multi-dataset evaluation, and systematic hyperparameter tuning, enabling future research on domain-specific language model training.

\section{Thesis Organization}

The remainder of this thesis is organized as follows:

\textbf{Chapter 2: Background and Related Work} reviews existing literature on financial NLP, language model pretraining objectives, data mixture strategies, and domain adaptation approaches. We position our work within the broader context of transfer learning and scaling laws research.

\textbf{Chapter 3: Methodology} describes our experimental design in detail, including model architecture (Qwen3 family), dataset characteristics (7 financial datasets totaling 207M tokens, plus WikiText), mixture strategies (50cap algorithm), and training setup. We document the iterative process of discovering and resolving learning rate sensitivity issues, demonstrating the scientific rigor underlying our empirical findings.

\textbf{Chapter 4: Results} presents experimental findings organized by theme, supported by visual evidence (11 scaling figures and 18 detailed tables). We begin with data mixture effects (the core finding), move to individual dataset analysis (component effects), examine training dynamics and learning-rate sensitivity, and conclude with domain transfer patterns. Scaling figures show performance trends across model sizes, while cross-dataset tables identify which training approaches perform best for each evaluation scenario.

\textbf{Chapter 5: Discussion} interprets our findings in light of existing theory and practice, leveraging the visual evidence from Chapter 4. We explain why WikiText underperforms on financial tasks (analyzing cross-dataset table boldface patterns), analyze the benefits of in-domain diversity (interpreting scaling figure trends), discuss practical aspects of learning rate sensitivity (connecting LR adjustment figures to observed stability), and provide concrete guidelines for practitioners training financial language models (supported by specific figure and table references).

\textbf{Chapter 6: Conclusion} summarizes contributions, discusses implications for research and practice, and outlines promising directions for future work, including extension to larger models, exploration of dynamic mixing strategies, and evaluation on downstream financial tasks.

\section{Scope and Limitations}

This thesis focuses specifically on pretraining dynamics for causal language models in the 0.6B-4B parameter range applied to financial text. Several important scope limitations should be noted:

\textbf{Model Architecture:} All experiments use the Qwen3 model family. While we believe our findings on learning rate scaling and data mixture effects are generalizable, validation on other architectures (LLaMA, Gemma, Phi) would strengthen confidence in universality.

\textbf{Data Mixture Strategy:} We use a single mixture algorithm (50cap, which caps the largest dataset at 50\% of the mixture). We did not explore square-root sampling, temperature-based sampling, or dynamic curriculum learning; these might lead to different results.

\textbf{Evaluation Methodology:} We evaluate models by perplexity on held-out test sets from the pretraining distribution. While perplexity often correlates with downstream performance, we do not directly measure accuracy on financial NLP tasks (sentiment, NER, QA). This is because we focus on pretraining dynamics, not downstream systems. But it limits how far we can generalize to applications.

\textbf{Scale Range:} Our experiments cover 0.6B to 4B parameters due to hardware limits. Larger models (7B+) may show different training dynamics and data sensitivity. Still, the range we study is relevant for edge deployment.

\textbf{Domain Specificity:} We focus on financial text. Many findings—especially about learning rate and dataset size—likely transfer to other domains, but the claim that WikiText helps little is domain-specific and may not hold elsewhere.

Despite these limits, our experiments across 30 models and 240+ evaluations provide solid evidence for the claims here. We try to separate clearly what we know from what needs more study.



\chapter{Background and Related Work}

This chapter reviews four areas that matter for our study. First, financial NLP. Then pretraining basics. Next, data-mixing strategies. Finally, domain adaptation and transfer learning. The aim is context, not a full survey.

\section{Financial NLP}

\subsection{The Financial NLP Landscape}

Financial NLP covers many tasks. Sentiment analysis on news and social media. Question answering on regulatory text. Numerical reasoning in reports. Information extraction from SEC filings \parencite{araci2019finbert, chen2021finqa}. The domain brings specific challenges that differ from general NLP: specialized vocabulary (e.g., "alpha", "beta", "EBITDA"), domain-specific reasoning (e.g., causal chains in market analysis), numerical grounding (reading financial statements), and temporal dynamics (events, earnings releases) \parencite{wu2023bloomberggpt, araci2019finbert}.

\subsection{Existing Financial Language Models}

Several finance-specialized LMs have appeared. \textbf{BloombergGPT} \parencite{wu2023bloomberggpt} is a 50B model trained on a 51\%/49\% mix of financial and general data. It scores well on financial benchmarks while keeping general ability. \textbf{FinBERT} variants \parencite{araci2019finbert, yang2020finbert} continue pretraining BERT on financial corpora and improve sentiment analysis on news. More recently, \textbf{FinGPT} \parencite{yang2023fingpt} explored open-source, instruction-tuned approaches for finance.

\subsection{Domain-Specific Challenges}

Three challenges keep coming up. \textbf{First}, privacy: institutions cannot upload sensitive data (portfolios, strategies, client information) to external APIs, so models must run locally \parencite{wu2023bloomberggpt}. \textbf{Second}, data scarcity: curated financial corpora are much smaller than general web text, so we need data-efficient training. \textbf{Third}, fast vocabulary change: terms like "DeFi" and "ESG" appear and shift with markets; models must adapt.

\section{Language Model Pretraining}

\subsection{Pretraining Objectives and Architecture}

Most models use the \textbf{causal language modeling} objective: predict the next token from the previous context \parencite{radford2019language, brown2020language}. It is self-supervised and scales to unlabeled corpora. Simple idea. Powerful in practice. Architecturally, decoder-only transformers (GPT, LLaMA, Qwen) dominate. Multi-head self-attention captures long-range dependencies; feed-forward layers add non-linearity \parencite{vaswani2017attention, touvron2023llama}.

\subsection{Scaling Laws and Model Size Effects}

\textcite{kaplan2020scaling} showed power-law relationships among model size, dataset size, compute, and performance. Larger models are more sample-efficient. That finding pushed the field toward billion-parameter models. Still, details matter. Later work added nuance. \textcite{hoffmann2022training} argued many models are undertrained relative to size (the Chinchilla view). \textcite{tay2022ul2} showed that objectives and data quality matter a lot for scaling.

\textbf{Hyperparameter sensitivity} gets less attention. \textcite{mccandlish2018empirical} noted that optimal learning rates can fall with model size. For 0.6B--4B models in specialized domains, systematic studies are limited. Many scaling-law papers assume "proper tuning" without saying how, which hides the messy part we examine empirically. Tuning matters at this scale. In our work, all main runs used LR=2e-5. In a few cases we reduced LR to stabilize training. We do not claim a general rule.

\subsection{Computational and Memory Considerations}

Training large language models takes real compute. A 1B-parameter model in 32-bit uses about 4GB just for parameters; optimizer states can double or triple that \parencite{rajbhandari2020zero,kingma2014adam}. For 0.6B--4B models, memory-savvy tricks help: mixed precision (bfloat16), gradient accumulation, activation checkpointing, and parameter-efficient methods like LoRA. Otherwise it does not fit. These make training feasible on enterprise GPUs (e.g., RTX A6000 48GB, A100 40GB, H100 80GB) \parencite{narayanan2021efficient,hu2021lora}.

\section{Data Mixture Strategies}

\subsection{Curriculum Learning and Sequential Mixing}

\textbf{Curriculum learning} sequences data from easier to harder, or from general to specialized \parencite{bengio2009curriculum}. \textcite{wu2022opt} used curriculum in OPT pretraining by increasing difficulty over time. In finance, a natural path is Wikipedia -> news -> SEC filings. Evidence is mixed at large scale \parencite{longpre2023pretrainer}. Some works report limited gains for masked LM; others see gains in narrow settings. Not universal. In practice, many systems sample from mixtures instead of strict curricula \parencite{raffel2020exploring,wu2022opt}.

\subsection{Simultaneous Mixture Approaches}

Another option is \textbf{simultaneous mixture}: sample from multiple datasets throughout training. \textcite{raffel2020exploring} (T5) used a multi-task mixture with task prefixes; diverse pretraining improved downstream generalization. \textcite{xie2023doremi} proposed DoReMi to adjust domain weights during training using validation perplexity, beating static mixtures on The Pile. Common in practice.

\textbf{BloombergGPT} \parencite{wu2023bloomberggpt} mixed 51\% financial with 49\% general data (The Pile, C4) at the token level. The balance kept general skills and added domain strength. But that study used one 50B model. How mixture and size interact (0.6B vs 4B) is less explored. We test this across three sizes. Mixed financial datasets (21.55 ppl @ 4B) clearly beat Wiki+Financial mixtures (26.69 ppl @ 4B; about 24\% worse), as shown in \Cref{fig:scaling_comparison_all,tab:mixed_financial_results,tab:mixed_wiki_financial_results}. For specialized use, domain purity can win over balance.

\subsection{Domain Proportions and Sampling Strategies}

Choosing domain proportions is not trivial. No single rule. Three strategies are common:

\textbf{1. Temperature sampling} \parencite{arivazhagan2019massively}: Sample from dataset $d$ with probability $p_d \propto n_d^{1/T}$ where $n_d$ is dataset size and $T$ is temperature. $T < 1$ upsamples small datasets; $T > 1$ downsamples them.

\textbf{2. Capping strategies} \parencite{longpre2023pretrainer}: Cap the largest dataset(s) at a threshold (e.g., 50\% of total tokens) to prevent dominance, then proportionally sample others. This ensures diversity even when one dataset is orders of magnitude larger.

\textbf{3. Equal mixing} \parencite{sanh2022multitask}: Assign equal sampling probability to each dataset regardless of size. This maximizes task diversity but may undersample large datasets.

We use a \textbf{50\% capping strategy} (``50cap'') for financial mixtures (details in Chapter 3) to balance diversity and data efficiency.

\section{Domain Adaptation and Transfer Learning}

\subsection{Cross-Domain Transfer in Language Models}

\textbf{Transfer learning} -- pretrain on broad data, then fine-tune for a domain -- has been standard since BERT \parencite{devlin2019bert,pan2010transfer,zhuang2020comprehensive}. The assumption is that general knowledge transfers. Often true. Not always. Recent work adds nuance. \textcite{gururangan2020don} showed \textbf{domain-adaptive pretraining} (continued pretraining on domain corpora) improves performance in biomedicine, computer science, news, and reviews. General pretraining alone is not enough for specialized use.

In finance, \textcite{araci2019finbert} improved results via continued pretraining on financial news; \textcite{yang2020finbert} added task-adaptive pretraining. \textcite{huang2023finbert} found domain-specific pretraining beats general models on financial IE. These are mostly BERT-style and classification-focused; domain adaptation for \textit{causal, generative} LMs in finance is less studied. Parameter-efficient methods (e.g., surgical fine-tuning \parencite{lee2022surgical}) hint that selective adaptation can help transfer and reduce forgetting.

\subsection{Catastrophic Forgetting and Stability}

\textbf{Catastrophic forgetting} is a classic issue: training further on a domain can erase general knowledge \parencite{mccloskey1989catastrophic, french1999catastrophic}. \textcite{kirkpatrick2017overcoming} proposed Elastic Weight Consolidation (EWC) to protect important parameters. With mixtures, \textit{simultaneous mixing} of general and domain data acts like implicit regularization by keeping the model exposed to diverse distributions \parencite{arivazhagan2019massively,raffel2020exploring}.

\subsection{Distribution Shift and Domain Mismatch}

\textbf{Distribution shift}—differences between training and evaluation—hurts generalization \parencite{quinonero2009dataset}. In finance this shows up as vocabulary shift (financial terms vs general language), discourse differences (analyst reports vs encyclopedic text), and formatting (tables in 10‑K vs narrative news). That hurts. \textcite{aharoni2020unsupervised} showed domain mismatch strongly degrades out‑of‑distribution performance. That motivates diverse mixtures that cover sub‑domains.

We test this directly. Does pretraining on high‑quality general text (WikiText) transfer to financial evaluation sets? Or does domain mismatch require in‑domain pretraining? And when we mix in‑domain datasets (sentiment, Q\&A, news, reports), do models generalize better than training on just one?

\subsection{Related Empirical Studies}

Several studies guide our setup. \textcite{xie2023doremi} showed dynamic mixture optimization can beat static mixes on The Pile, but it needs validation data and multiple runs. Useful, but not always practical. \textcite{longpre2023pretrainer} surveyed practitioners and found capping and temperature sampling common in production. \textcite{mitra2023orca2} (Orca‑2) showed that diverse instruction formats help reasoning generalization, suggesting \textit{intra‑domain diversity} (multiple financial datasets) can matter as much as domain specialization.

What is missing is a systematic look at \textbf{dataset size effects} for mixtures. When is a dataset large enough for standalone pretraining? When does mixing help, and when does it hurt? How do these patterns change with model size? These questions shape our experimental design in Chapter 3.

\chapter{Methodology}

We describe the experimental design, models, datasets, training setup, and evaluation protocol used to study data mixture effects in financial LM pretraining.

\section{Experimental Design}
We run 10 pretraining configurations (mixtures and single sources) at three model sizes (0.6B, 1.7B, 4B), yielding 30 models and 240 evaluations over eight test sets. Experiments isolate impacts of (i) mixture composition, (ii) model size and learning rate scaling, and (iii) dataset size and format.

\section{Models}
We use the Qwen2 family of decoder-only transformers \parencite{yang2024qwen2}, chosen for architectural consistency across sizes and efficient inference. We train 0.6B, 1.7B, and 4B models with grouped-query attention and bfloat16 mixed precision.

\section{Datasets and Mixtures}
Financial sources include seven datasets covering long-form documents (News: 197M tokens; SEC: 80M), instruction formats (FinGPT: 19M; Alpaca: 17M), short-form Q\&A (FiQA: 4M; Financial QA: 3.5M), and micro-text (Twitter: 0.3M). General data is WikiText-103 (100M tokens) \parencite{merity2016pointer}. We build (i) Mixed Financial with 50\% capping to prevent dominance, (ii) Mixed Wiki+Financial, and (iii) the seven single-source runs.

\section{Training Setup}
We use causal LM pretraining with Adam-family optimizer, global batch size selected per model to process $\sim$100M tokens per run, gradient accumulation for memory fit, and activation checkpointing. Learning rate follows a cosine schedule with warmup; crucially, we reduce LR with model size (empirically close to $1/\sqrt{N}$), which resolves reverse scaling observed under a constant LR \parencite{mccandlish2018empirical}. Implementation uses ZeRO-style sharding or equivalent memory optimization \parencite{rajbhandari2020zero}.

\section{Evaluation Protocol and Metrics}
Each model is evaluated on eight held-out test sets (seven financial + WikiText). We report:
\begin{itemize}
  \item Cross-entropy loss: $\mathcal{L} = -\frac{1}{N}\sum_i \log P(w_i\mid w_{<i})$.
  \item Perplexity: $\mathrm{PPL} = \exp(\mathcal{L})$.
  \item Coefficient of Variation (CV): robustness across datasets. Let $\mathbf{p} = [\mathrm{PPL}_d]_{d\in\mathcal{D}}$ be perplexities on the eight test sets; with macro averaging across datasets,
  $\mu = \frac{1}{|\mathcal{D}|}\sum_d \mathrm{PPL}_d$, $\sigma = \sqrt{\frac{1}{|\mathcal{D}|-1}\sum_d (\mathrm{PPL}_d - \mu)^2}$, and $\mathrm{CV}\% = 100\,\sigma/\mu$.
\end{itemize}
We exclude non-finite values from CV and flag such runs in tables. In-domain CV (within a dataset’s subdivisions) is computed analogously; cross-dataset CV aggregates the eight-set vector.

\chapter{Results}

This chapter presents detailed findings while preserving all figures and tables. We expand on mixture effects, learning-rate sensitivity, dataset size and format, and cross-dataset transfer patterns.

\begin{table}[h]
\centering
\caption{Overview of 10 pretraining experiments. Per dataset, we pretrain at 0.6B/1.7B/4B and evaluate on 8 test sets. LR adjustments are applied where noted.}
\label{tab:experiments_overview}
\resizebox{\textwidth}{!}{
\begin{tabular}{l l l l}
\toprule
\textbf{Experiment} & \textbf{Training source} & \textbf{Tokens} & \textbf{Notes} \\
\midrule
Mixed Financial & 7 financial datasets & 207M & 50\% capping (50cap) \;\; strong financial performance \\
Mixed Wiki+Financial & WikiText + 7 financial & $\sim$400M & Improves WikiText; degrades financial vs Mixed Financial \\
WikiText & WikiText-103 & 100M & General-domain baseline; LR sensitive at scale \\
Financial News & News articles & 197M & Long-form; low CV; good standalone \\
SEC Reports & Regulatory filings & 80M & Long-form; low CV; good standalone \\
FinGPT & Instruction mixture & 19M & Instruction format cluster \\
Alpaca (Finance) & Instruction mixture & 17M & Instruction format cluster \\
FiQA & Short Q\&A & 4M & Short-form; moderate CV \\
Financial QA 10K & Q\&A (10K examples) & 3.5M & Very small; high CV; LR tuning needed \\
Twitter Financial & Tweets & 0.3M & Very small; short-form outlier; highest CV \\
\bottomrule
\end{tabular}}
\end{table}

\section{Mixture Effects}
\textbf{Summary.} Mixed financial datasets outperform pure WikiText on all financial evaluations, and outperform Mixed Wiki+Financial when the objective is finance. Adding WikiText marginally improves general-domain performance but dilutes financial specialization.

\textbf{Evidence.} \Cref{fig:scaling_mixed_financial,fig:scaling_mixed_wiki_financial} visualize scaling across sizes; 4B Mixed Financial achieves 21.55 ppl (mean across financial sets), whereas Mixed Wiki+Financial degrades to 26.69 ppl despite gains on WikiText. \Cref{tab:mixed_financial_results,tab:mixed_wiki_financial_results} quantify per-dataset outcomes and highlight best-performing cells.

\begin{figure}[H]
  \centering
  \includegraphics[width=\textwidth]{../thesis/figures/scaling_mixed_financial.png}
  \caption{Mixed Financial scaling.}\label{fig:scaling_mixed_financial}
\end{figure}

\begin{figure}[H]
  \centering
  \includegraphics[width=\textwidth]{../thesis/figures/scaling_mixed_wiki_financial.png}
  \caption{Mixed Wiki+Financial scaling.}\label{fig:scaling_mixed_wiki_financial}
\end{figure}

\section{Scaling and LR Sensitivity}
\textbf{Reverse scaling and fix.} With a constant LR, 1.7B/4B sometimes underperform 0.6B (``reverse scaling''). Adjusting LR by size resolves this. Empirically, reducing LR roughly with $1/\sqrt{N}$ restores expected ordering and improves 10--32\%.

\textbf{Evidence.} \Cref{fig:scaling_wikitext,fig:scaling_financial_qa,fig:scaling_twitter} compare original vs adjusted LRs (solid vs dashed). Tables \Cref{tab:wikitext_lr_comparison,tab:financial_qa_lr_comparison,tab:twitter_lr_comparison} show per-dataset improvements under the tuned LR.

\begin{figure}[H]
  \centering
  \includegraphics[width=\textwidth]{../thesis/figures/scaling_wikitext.png}
  \caption{WikiText LR comparison.}\label{fig:scaling_wikitext}
\end{figure}

\begin{figure}[H]
  \centering
  \includegraphics[width=\textwidth]{../thesis/figures/scaling_financial_qa.png}
  \caption{Financial QA: LR adjustment resolves reverse scaling.}\label{fig:scaling_financial_qa}
\end{figure}

\begin{figure}[H]
  \centering
  \includegraphics[width=\textwidth]{../thesis/figures/scaling_twitter.png}
  \caption{Twitter: severe LR sensitivity at small data scales.}\label{fig:scaling_twitter}
\end{figure}

\section{Dataset Size and Format}
\textbf{Size thresholds.} Large datasets (News: 197M tokens; SEC: 80M) sustain standalone pretraining with low variance (26--32\% CV). Small datasets (Financial QA: 3.5M; Twitter: 0.3M) severely overtrain (tens to hundreds of epochs) and exhibit high variance (up to 89\% CV), motivating mixtures.

\textbf{Format matters.} Transfer depends strongly on format: long-form document models (News, SEC) transfer across each other better than to short-form (Twitter) or instruction formats (FinGPT/Alpaca); instruction-tuned sources cluster; short-form Twitter remains an outlier. Figures \Cref{fig:scaling_news_articles,fig:scaling_sec_reports,fig:scaling_fingpt,fig:scaling_alpaca,fig:scaling_fiqa} illustrate scaling within format families.

\begin{figure}[H]
  \centering
  \includegraphics[width=\textwidth]{../thesis/figures/scaling_news_articles.png}
  \caption{News Articles scaling.}\label{fig:scaling_news_articles}
\end{figure}

\begin{figure}[H]
  \centering
  \includegraphics[width=\textwidth]{../thesis/figures/scaling_sec_reports.png}
  \caption{SEC Reports scaling.}\label{fig:scaling_sec_reports}
\end{figure}

\begin{figure}[H]
  \centering
  \includegraphics[width=\textwidth]{../thesis/figures/scaling_fingpt.png}
  \caption{FinGPT instruction mixture scaling.}\label{fig:scaling_fingpt}
\end{figure}

\begin{figure}[H]
  \centering
  \includegraphics[width=\textwidth]{../thesis/figures/scaling_alpaca.png}
  \caption{Alpaca instruction mixture scaling.}\label{fig:scaling_alpaca}
\end{figure}

\begin{figure}[H]
  \centering
  \includegraphics[width=\textwidth]{../thesis/figures/scaling_fiqa.png}
  \caption{FiQA short-form scaling.}\label{fig:scaling_fiqa}
\end{figure}

\begin{figure}[H]
  \centering
  \includegraphics[width=\textwidth]{../thesis/figures/scaling_comparison_all.png}
  \caption{Comparison across training sources.}\label{fig:scaling_comparison_all}
\end{figure}

\section{All Tables (Preserved)}
We include all result tables for completeness; boldface indicates best values along the specified axis (row-wise minima for results tables, pair-wise minima for LR comparisons, and column-wise best for cross-dataset tables).

% Results by training dataset
% Mixed Financial Dataset: Evaluation Results
% Training: Mixed Financial (7 datasets mixed, 322M tokens)
% All models trained with LR=2e-5

\begin{table}[h]
\centering
\caption{Mixed Financial Dataset: Evaluation Across Multiple Datasets}
\label{tab:mixed_financial_results}
\resizebox{\textwidth}{!}{
\begin{tabular}{l|ccc|ccc}
\toprule
\multirow{2}{*}{\textbf{Eval Dataset}} &
\multicolumn{3}{c|}{\textbf{Cross-Entropy Loss}} &
\multicolumn{3}{c}{\textbf{Perplexity}} \\
\cmidrule(lr){2-4} \cmidrule(lr){5-7}
 & \textbf{0.6B} & \textbf{1.7B} & \textbf{4B} & \textbf{0.6B} & \textbf{1.7B} & \textbf{4B}\\
\midrule
Alpaca & 4.54 & 3.38 & \textbf{2.97} & 93.35 & 29.53 & \textbf{19.50}\\
Financial News & 4.03 & 3.05 & \textbf{2.63} & 56.35 & 21.19 & \textbf{13.84}\\
Financial Qa & 5.21 & 3.75 & \textbf{3.23} & 183.7 & 42.30 & \textbf{25.14}\\
Financial Repor & 4.94 & 3.58 & \textbf{3.11} & 139.6 & 35.83 & \textbf{22.36}\\
Fingpt & 5.04 & 3.63 & \textbf{3.14} & 153.9 & 37.82 & \textbf{23.08}\\
Fiqa & 4.63 & 3.46 & \textbf{3.05} & 102.5 & 31.85 & \textbf{21.20}\\
Twitter & 5.21 & 3.76 & \textbf{3.25} & 182.6 & 42.91 & \textbf{25.72}\\
\bottomrule
\end{tabular}
}
\end{table}


% Mixed Wiki+Financial Dataset: Evaluation Results
% Training: Mixed Wiki+Financial (WikiText + 7 financial datasets, 343.35M tokens)
% All models trained with LR=2e-5

\begin{table}[h]
\centering
\caption[Mixed Wiki+Financial: Evaluation Results]{Mixed Wiki+Financial Dataset: Evaluation Across Multiple Datasets}
\label{tab:mixed_wiki_financial_results}
\begin{tabular}{l|ccc|ccc}
\hline
\textbf{Eval Dataset} & \multicolumn{3}{c|}{\textbf{Cross-Entropy Loss}} & \multicolumn{3}{c}{\textbf{Perplexity}} \\
\cline{2-4} \cline{5-7}
  & \textbf{0.6B} & \textbf{1.7B} & \textbf{4B} & \textbf{0.6B} & \textbf{1.7B} & \textbf{4B} \\
Alpaca & 4.07 & 3.48 & 3.15 & 58.56 & 32.38 & 23.23 \\
Financial News & 3.65 & 3.13 & 2.77 & 38.68 & 22.79 & 15.91 \\
Financial QA & 4.58 & 3.87 & 3.46 & 97.49 & 47.94 & 31.76 \\
SEC Reports & 4.35 & 3.69 & 3.33 & 77.57 & 40.17 & 27.91 \\
FinGPT & 4.44 & 3.75 & 3.37 & 84.43 & 42.50 & 28.92 \\
FiQA & 4.14 & 3.56 & 3.24 & 63.03 & 35.04 & 25.61 \\
Twitter & 4.59 & 3.88 & 3.48 & 98.13 & 48.42 & 32.48 \\
Wikitext & 4.41 & 3.74 & 3.32 & 82.10 & 41.95 & 27.72 \\
\hline
\textbf{Average} & \textbf{4.28} & \textbf{3.64} & \textbf{3.26} & \textbf{75.00} & \textbf{38.90} & \textbf{26.69} \\
\hline
\end{tabular}
\end{table}

% WikiText Dataset: Evaluation Results
% Training: WikiText (WikiText-103, 100M tokens)
% All models trained with LR=2e-5

\begin{table}[h]
\centering
\caption{WikiText Dataset: Evaluation Across Multiple Datasets}
\label{tab:wikitext_results}
\begin{tabular}{l|ccc|ccc}
\hline
\textbf{Eval Dataset} & \multicolumn{3}{c|}{\textbf{Cross-Entropy Loss}} & \multicolumn{3}{c}{\textbf{Perplexity}} \\n\cline{2-4} \cline{5-7}
  & \textbf{0.6B} & \textbf{1.7B} & \textbf{4B} & \textbf{0.6B} & \textbf{1.7B} & \textbf{4B} \\
Alpaca & \textbf{2.22} & 3.24 & 3.48 & \textbf{9.23} & 25.51 & 32.38 \
 Financial News & \textbf{2.62} & 2.93 & 3.37 & \textbf{13.70} & 18.78 & 29.19 \
 Financial Repor & \textbf{1.39} & 3.27 & 3.44 & \textbf{3.99} & 26.46 & 31.23 \
 Fingpt & \textbf{1.30} & 2.11 & 3.57 & \textbf{3.67} & 8.27 & 35.50 \
 Fiqa & \textbf{2.07} & 3.14 & 3.53 & \textbf{7.89} & 23.15 & 34.03 \
 Twitter & \textbf{1.45} & 2.78 & 3.52 & \textbf{4.26} & 16.06 & 33.71 \
\hline
\end{tabular}
\end{table}


% Financial News Dataset: Evaluation Results
% Training: Financial News (Financial news articles, 197M tokens)
% All models trained with LR=2e-5

\begin{table}[h]
\centering
\caption[Financial News: Evaluation Results]{Financial News Dataset: Evaluation Across Multiple Datasets}
\label{tab:news_articles_results}
\begin{tabular}{l|ccc|ccc}
\hline
\textbf{Eval Dataset} & \multicolumn{3}{c|}{\textbf{Cross-Entropy Loss}} & \multicolumn{3}{c}{\textbf{Perplexity}} \\
\cline{2-4} \cline{5-7}
  & \textbf{0.6B} & \textbf{1.7B} & \textbf{4B} & \textbf{0.6B} & \textbf{1.7B} & \textbf{4B} \\
Alpaca & 4.57 & 3.61 & \textbf{3.39} & 96.31 & \textbf{36.92} & \textbf{29.75} \\
Financial Qa & 5.11 & 3.90 & \textbf{3.66} & 166.1 & \textbf{49.53} & \textbf{38.90} \\
Financial Repor & 4.85 & 3.73 & \textbf{3.51} & 127.7 & \textbf{41.68} & \textbf{33.46} \\
Fingpt & 5.08 & 3.90 & \textbf{3.64} & 160.9 & \textbf{49.56} & \textbf{38.03} \\
Fiqa & 4.62 & 3.65 & \textbf{3.46} & 101.3 & \textbf{38.68} & \textbf{31.69} \\
Twitter & 5.11 & 3.91 & \textbf{3.66} & 165.2 & \textbf{49.88} & \textbf{38.98} \\
Wikitext & 4.95 & 3.81 & \textbf{3.54} & 140.7 & \textbf{45.17} & \textbf{34.33} \\
\hline
\end{tabular}
\end{table}


% SEC Reports Dataset: Evaluation Results
% Training: SEC Reports (SEC 10-K/10-Q filings, 80M tokens)
% All models trained with LR=2e-5

\begin{table}[h]
\centering
\caption{SEC Reports Dataset: Evaluation Across Multiple Datasets}
\label{tab:sec_reports_results}
\begin{tabular}{l|ccc|ccc}
\hline
\textbf{Eval Dataset} & \multicolumn{3}{c|}{\textbf{Cross-Entropy Loss}} & \multicolumn{3}{c}{\textbf{Perplexity}} \\n\cline{2-4} \cline{5-7}
  & \textbf{0.6B} & \textbf{1.7B} & \textbf{4B} & \textbf{0.6B} & \textbf{1.7B} & \textbf{4B} \\
Alpaca & 3.86 & 3.14 & \textbf{2.92} & 47.65 & \textbf{23.04} & \textbf{18.54} \
 Financial News & 3.71 & 3.08 & \textbf{2.81} & 40.85 & \textbf{21.65} & \textbf{16.67} \
 Financial Qa & 3.90 & 3.08 & \textbf{2.86} & 49.30 & \textbf{21.77} & \textbf{17.39} \
 Fingpt & 3.97 & 3.15 & \textbf{2.93} & 53.18 & \textbf{23.41} & \textbf{18.68} \
 Fiqa & 3.85 & 3.14 & \textbf{2.96} & 47.22 & \textbf{23.15} & \textbf{19.34} \
 Twitter & 3.94 & 3.13 & \textbf{2.90} & 51.30 & \textbf{22.86} & \textbf{18.12} \
 Wikitext & 3.89 & 3.10 & \textbf{2.88} & 49.02 & \textbf{22.21} & \textbf{17.72} \
\hline
\end{tabular}
\end{table}


% FinGPT Sentiment Dataset: Evaluation Results
% Training: FinGPT Sentiment (FinGPT/fingpt-sentiment-train, 19M tokens)
% All models trained with LR=2e-5

\begin{table}[h]
\centering
\caption{FinGPT Sentiment Dataset: Evaluation Across Multiple Datasets}
\label{tab:fingpt_results}
\begin{tabular}{l|ccc|ccc}
\hline
\textbf{Eval Dataset} & \multicolumn{3}{c|}{\textbf{Cross-Entropy Loss}} & \multicolumn{3}{c}{\textbf{Perplexity}} \\n\cline{2-4} \cline{5-7}
  & \textbf{0.6B} & \textbf{1.7B} & \textbf{4B} & \textbf{0.6B} & \textbf{1.7B} & \textbf{4B} \\
Alpaca & 3.57 & 2.55 & \textbf{2.11} & 35.55 & \textbf{12.78} & \textbf{8.27} \
 Financial News & 3.36 & 2.45 & \textbf{2.07} & 28.72 & \textbf{11.58} & \textbf{7.92} \
 Financial Qa & 3.66 & 2.38 & \textbf{1.83} & 38.96 & \textbf{10.85} & \textbf{6.24} \
 Financial Repor & 3.53 & 2.31 & \textbf{1.82} & 33.97 & \textbf{10.12} & \textbf{6.20} \
 Fiqa & 3.57 & 2.55 & \textbf{2.10} & 35.64 & \textbf{12.79} & \textbf{8.16} \
 Twitter & 3.68 & 2.40 & \textbf{1.87} & 39.54 & \textbf{11.05} & \textbf{6.46} \
 Wikitext & 3.66 & 2.44 & \textbf{1.99} & 38.70 & \textbf{11.46} & \textbf{7.29} \
\hline
\end{tabular}
\end{table}


% Finance Alpaca Dataset: Evaluation Results
% Training: Finance Alpaca (gbharti/finance-alpaca, 17M tokens)
% All models trained with LR=2e-5

\begin{table}[h]
\centering
\caption{Finance Alpaca Dataset: Evaluation Across Multiple Datasets}
\label{tab:alpaca_results}
\resizebox{\textwidth}{!}{
\begin{tabular}{l|ccc|ccc}
\toprule
\multirow{2}{*}{\textbf{Eval Dataset}} &
\multicolumn{3}{c|}{\textbf{Cross-Entropy Loss}} &
\multicolumn{3}{c}{\textbf{Perplexity}} \\
\cmidrule(lr){2-4} \cmidrule(lr){5-7}
 & \textbf{0.6B} & \textbf{1.7B} & \textbf{4B} & \textbf{0.6B} & \textbf{1.7B} & \textbf{4B}\\
\midrule
Financial News & 3.92 & 2.71 & \textbf{2.15} & 50.40 & 15.05 & \textbf{8.58}\\
Financial Qa & 4.77 & 2.95 & \textbf{2.15} & 117.4 & 19.11 & \textbf{8.56}\\
Financial Repor & 4.54 & 2.85 & \textbf{2.11} & 93.56 & 17.26 & \textbf{8.25}\\
Fingpt & 4.71 & 2.99 & \textbf{2.22} & 111.7 & 19.85 & \textbf{9.18}\\
Fiqa & 4.29 & 2.87 & \textbf{2.22} & 73.12 & 17.63 & \textbf{9.22}\\
Twitter & 4.78 & 2.99 & \textbf{2.19} & 118.7 & 19.82 & \textbf{8.97}\\
Wikitext & 4.63 & 2.94 & \textbf{2.18} & 102.4 & 18.85 & \textbf{8.88}\\
\bottomrule
\end{tabular}
}
\end{table}


% FiQA Dataset: Evaluation Results
% Training: FiQA (FiQA dataset, 4M tokens)
% All models trained with LR=2e-5

\begin{table}[h]
\centering
\caption[FiQA: Evaluation Results]{FiQA Dataset: Evaluation Across Multiple Datasets}
\label{tab:fiqa_results}
\begin{tabular}{l|ccc|ccc}
\hline
\textbf{Eval Dataset} & \multicolumn{3}{c|}{\textbf{Cross-Entropy Loss}} & \multicolumn{3}{c}{\textbf{Perplexity}} \\
\cline{2-4} \cline{5-7}
  & \textbf{0.6B} & \textbf{1.7B} & \textbf{4B} & \textbf{0.6B} & \textbf{1.7B} & \textbf{4B} \\
Alpaca & 4.14 & 2.56 & 1.96 & 62.97 & 12.96 & 7.12 \\
Financial News & 3.90 & 2.54 & 2.01 & 49.22 & 12.74 & 7.43 \\
Financial QA & 4.64 & 2.60 & 1.84 & 103.4 & 13.53 & 6.32 \\
SEC Reports & 4.42 & 2.53 & 1.81 & 83.48 & 12.51 & 6.14 \\
\textbf{FiQA} & \textbf{4.17} & \textbf{2.56} & \textbf{1.96} & \textbf{64.75} & \textbf{12.99} & \textbf{7.08} \\
FinGPT & 4.67 & 2.71 & 1.95 & 107.2 & 15.08 & 7.01 \\
Twitter & 4.66 & 2.65 & 1.88 & 105.3 & 14.10 & 6.58 \\
Wikitext & 4.52 & 2.63 & 1.91 & 92.13 & 13.81 & 6.72 \\
\hline
\textbf{Average} & \textbf{4.39} & \textbf{2.60} & \textbf{1.92} & \textbf{83.57} & \textbf{13.47} & \textbf{6.80} \\
\hline
\end{tabular}
\end{table}

% Twitter Financial Dataset: Evaluation Results
% Training: Twitter Financial (Financial tweets, 0.3M tokens)
% All models trained with LR=2e-5

\begin{table}[h]
\centering
\caption{Twitter Financial Dataset: Evaluation Across Multiple Datasets}
\label{tab:twitter_results}
\begin{tabular}{l|ccc|ccc}
\hline
\textbf{Eval Dataset} & \multicolumn{3}{c|}{\textbf{Cross-Entropy Loss}} & \multicolumn{3}{c}{\textbf{Perplexity}} \\n\cline{2-4} \cline{5-7}
  & \textbf{0.6B} & \textbf{1.7B} & \textbf{4B} & \textbf{0.6B} & \textbf{1.7B} & \textbf{4B} \\
Alpaca & 3.01 & \textbf{2.66} & 2.96 & 20.21 & \textbf{14.33} & 19.20 \
 Financial News & 3.17 & \textbf{2.80} & 2.87 & 23.77 & \textbf{16.48} & 17.67 \
 Financial Qa & 2.46 & \textbf{2.32} & 2.83 & 11.76 & \textbf{10.15} & 16.98 \
 Financial Repor & 2.48 & \textbf{2.32} & 2.80 & 11.95 & \textbf{10.17} & 16.42 \
 Fingpt & 2.74 & \textbf{2.50} & 2.91 & 15.53 & \textbf{12.23} & 18.34 \
 Fiqa & 2.98 & \textbf{2.66} & 3.00 & 19.67 & \textbf{14.26} & 20.09 \
 Wikitext & 2.69 & \textbf{2.47} & 2.88 & 14.74 & \textbf{11.78} & 17.85 \
\hline
\end{tabular}
\end{table}


% Financial QA 10K Dataset: Evaluation Results
% Training: Financial QA 10K (virattt/financial-qa-10K, 3.5M tokens)
% All models trained with LR=2e-5

\begin{table}[h]
\centering
\caption{Financial QA 10K Dataset: Evaluation Across Multiple Datasets}
\label{tab:financial_qa_results}
\resizebox{\textwidth}{!}{
\begin{tabular}{l|ccc|ccc}
\toprule
\multirow{2}{*}{\textbf{Eval Dataset}} &
\multicolumn{3}{c|}{\textbf{Cross-Entropy Loss}} &
\multicolumn{3}{c}{\textbf{Perplexity}} \\
\cmidrule(lr){2-4} \cmidrule(lr){5-7}
& \textbf{0.6B} & \textbf{1.7B} & \textbf{4B} & \textbf{0.6B} & \textbf{1.7B} & \textbf{4B} \\
\midrule
Alpaca & 2.38 & 2.23 & 2.29 & 10.82 & 9.31 & 9.91 \\
Financial News & 2.36 & 2.17 & 2.13 & 10.60 & 8.78 & 8.41 \\
Financial Repor & 2.11 & 2.00 & 2.11 & 8.21 & 7.40 & 8.25 \\
Fingpt & 2.31 & 2.15 & 2.23 & 10.04 & 8.62 & 9.34 \\
Fiqa & 2.40 & 2.25 & 2.31 & 11.02 & 9.45 & 10.05 \\
Twitter & 2.21 & 2.10 & 2.20 & 9.14 & 8.18 & 8.99 \\
Wikitext & 2.24 & 2.11 & 2.19 & 9.41 & 8.23 & 8.89 \\
\bottomrule
\end{tabular}
}
\end{table}



% LR comparisons
% WikiText Dataset: Evaluation Results with LR Adjustments
% Training: WikiText (WikiText-103, 100M tokens)
% LR Adjustments: 1.7B (2e-5 → 5e-6), 4B (2e-5 → 3e-6)

\begin{table}[h]
\centering
\caption[WikiText: Learning Rate Comparison]{WikiText Dataset: Impact of Learning Rate Adjustments}
\label{tab:wikitext_lr_comparison}
\begin{tabular}{l|c|cc|cc|c|cc|cc}
\hline
\multirow{3}{*}{\textbf{Eval Dataset}} &
\multicolumn{5}{c|}{\textbf{Cross-Entropy Loss}} &
\multicolumn{5}{c}{\textbf{Perplexity}} \\
\cline{2-6} \cline{7-11}
& \textbf{0.6B} & \multicolumn{2}{c|}{\textbf{1.7B}} & \multicolumn{2}{c|}{\textbf{4B}} &
 \textbf{0.6B} & \multicolumn{2}{c|}{\textbf{1.7B}} & \multicolumn{2}{c}{\textbf{4B}} \\
\cline{3-4} \cline{5-6} \cline{8-9} \cline{10-11}
& \textbf{2e-5} & \textbf{2e-5} & \textbf{5e-6} & \textbf{2e-5} & \textbf{3e-6} &
 \textbf{2e-5} & \textbf{2e-5} & \textbf{5e-6} & \textbf{2e-5} & \textbf{3e-6} \\
\hline
 Alpaca & 2.22 & \textbf{3.24} & 3.79 & \textbf{3.48} & 3.64 & 9.23 & \textbf{25.51} & 44.22 & \textbf{32.38} & 38.06 \\
Financial News & 2.62 & \textbf{2.93} & 3.52 & 3.37 & \textbf{3.27} & 13.70 & \textbf{18.78} & 33.66 & \textbf{29.19} & \textbf{26.44} \\
 Financial QA & 3.40 & 10.67 & \textbf{4.07} & \textbf{3.37} & 3.87 & 29.90 & $\infty$ & \textbf{58.33} & \textbf{29.08} & 47.98 \\
 SEC Reports & 1.39 & \textbf{3.27} & 3.91 & \textbf{3.44} & 3.75 & 3.99 & \textbf{26.46} & 49.83 & \textbf{31.23} & 42.41 \\
 FinGPT & 1.30 & \textbf{2.11} & 4.07 & \textbf{3.57} & 3.88 & 3.67 & \textbf{8.27} & 58.55 & \textbf{35.50} & 48.30 \\
 FiQA & 2.07 & \textbf{3.14} & 3.85 & \textbf{3.53} & 3.74 & 7.89 & \textbf{23.15} & 46.81 & \textbf{34.03} & 42.04 \\
Twitter & 1.45 & \textbf{2.78} & 4.08 & \textbf{3.52} & 3.88 & 4.26 & \textbf{16.06} & 58.98 & \textbf{33.71} & 48.48 \\
\rowcolor{gray!20} \textbf{Wikitext (train)} & 1.56 & \textbf{3.42} & 3.88 & \textbf{3.30} & 3.65 & 4.78 & \textbf{30.63} & 48.44 & \textbf{27.19} & 38.60 \\
\rowcolor{blue!10} \textbf{Average} & \textbf{2.00} & \textbf{3.95} & \textbf{3.89} & \textbf{3.45} & \textbf{3.71} & \textbf{9.68} & \textbf{$\infty$} & \textbf{49.85} & \textbf{31.54} & \textbf{41.54}  \\
\hline
\end{tabular}
\end{table}

% Twitter Financial Dataset: Evaluation Results with LR Adjustments
% Training: Twitter Financial (Financial tweets, 0.3M tokens)
% LR Adjustments: 1.7B (2e-5 → 1e-5), 4B (2e-5 → 5e-6)

\begin{table}[h]
\centering
\caption{Twitter Financial Dataset: Impact of Learning Rate Adjustments}
\label{tab:twitter_lr_comparison}
\begin{tabular}{l|c|cc|cc|c|cc|cc}
\hline
\multirow{3}{*}{\textbf{Eval Dataset}} &
\multicolumn{5}{c|}{\textbf{Cross-Entropy Loss}} &
\multicolumn{5}{c}{\textbf{Perplexity}} \\
\cline{2-6} \cline{7-11}
& \textbf{0.6B} & \multicolumn{2}{c|}{\textbf{1.7B}} & \multicolumn{2}{c|}{\textbf{4B}} &
 \textbf{0.6B} & \multicolumn{2}{c|}{\textbf{1.7B}} & \multicolumn{2}{c}{\textbf{4B}} \
\cline{3-4} \cline{5-6} \cline{8-9} \cline{10-11}
& \textbf{2e-5} & \textbf{2e-5} & \textbf{1e-5} & \textbf{2e-5} & \textbf{5e-6} &
 \textbf{2e-5} & \textbf{2e-5} & \textbf{1e-5} & \textbf{2e-5} & \textbf{5e-6} \
\hline
 Alpaca & 3.01 & 2.66 & \textbf{2.54} & 2.96 & \textbf{2.61} & 20.21 & 14.33 & \textbf{12.66} & \textbf{19.20} & \textbf{13.65} \
 Financial News & 3.17 & 2.80 & \textbf{2.65} & 2.87 & \textbf{2.54} & 23.77 & 16.48 & \textbf{14.10} & \textbf{17.67} & \textbf{12.68} \
 Financial Qa & 2.46 & 2.32 & \textbf{2.16} & 2.83 & \textbf{2.43} & 11.76 & 10.15 & \textbf{8.69} & \textbf{16.98} & \textbf{11.39} \
 Financial Repor & 2.48 & 2.32 & \textbf{2.16} & 2.80 & \textbf{2.39} & 11.95 & 10.17 & \textbf{8.70} & \textbf{16.42} & \textbf{10.93} \
 Fingpt & 2.74 & 2.50 & \textbf{2.34} & 2.91 & \textbf{2.54} & 15.53 & 12.23 & \textbf{10.41} & \textbf{18.34} & \textbf{12.69} \
 Fiqa & 2.98 & 2.66 & \textbf{2.50} & 3.00 & \textbf{2.61} & 19.67 & 14.26 & \textbf{12.20} & \textbf{20.09} & \textbf{13.61} \
\rowcolor{gray!20} \textbf{Twitter (train)} & 2.53 & 2.40 & \textbf{2.22} & 2.88 & \textbf{2.47} & 12.60 & 11.02 & \textbf{9.21} & 17.83 & \textbf{11.81} \
 Wikitext & 2.69 & 2.47 & \textbf{2.30} & 2.88 & \textbf{2.49} & 14.74 & 11.78 & \textbf{9.94} & \textbf{17.85} & \textbf{12.02} \
\rowcolor{blue!10} \textbf{Average} & \textbf{2.76} & \textbf{2.52} & \textbf{2.36} & \textbf{2.89} & \textbf{2.51} & \textbf{16.28} & \textbf{12.55} & \textbf{10.74} & \textbf{18.05} & \textbf{12.35}  \
\hline
\end{tabular}
\end{table}


% Financial QA 10K Dataset: Evaluation Results with LR Adjustments
% Training: Financial QA 10K (virattt/financial-qa-10K, 3.5M tokens)
% LR Adjustments: 1.7B (2e-5 → 1e-5), 4B (2e-5 → 5e-6)

\begin{table}[h]
\centering
\caption{Financial QA 10K Dataset: Impact of Learning Rate Adjustments}
\label{tab:financial_qa_lr_comparison}
\begin{tabular}{l|c|cc|cc|c|cc|cc}
\hline
\multirow{3}{*}{\textbf{Eval Dataset}} &
\multicolumn{5}{c|}{\textbf{Cross-Entropy Loss}} &
\multicolumn{5}{c}{\textbf{Perplexity}} \\
\cline{2-6} \cline{7-11}
& \textbf{0.6B} & \multicolumn{2}{c|}{\textbf{1.7B}} & \multicolumn{2}{c|}{\textbf{4B}} &
 \textbf{0.6B} & \multicolumn{2}{c|}{\textbf{1.7B}} & \multicolumn{2}{c}{\textbf{4B}} \
\cline{3-4} \cline{5-6} \cline{8-9} \cline{10-11}
& \textbf{2e-5} & \textbf{2e-5} & \textbf{1e-5} & \textbf{2e-5} & \textbf{5e-6} &
 \textbf{2e-5} & \textbf{2e-5} & \textbf{1e-5} & \textbf{2e-5} & \textbf{5e-6} \
\hline
 Alpaca & 2.38 & \textbf{2.23} & 2.29 & 2.29 & \textbf{2.18} & 10.82 & \textbf{9.31} & 9.92 & \textbf{9.91} & \textbf{8.88} \
 Financial News & 2.36 & \textbf{2.17} & 2.23 & 2.13 & \textbf{2.04} & 10.60 & \textbf{8.78} & 9.25 & \textbf{8.41} & \textbf{7.71} \
\rowcolor{gray!20} \textbf{Financial Qa (train)} & 2.12 & \textbf{2.01} & 2.12 & 2.12 & \textbf{2.01} & 8.29 & \textbf{7.44} & 8.29 & 8.29 & \textbf{7.43} \
 Financial Repor & 2.11 & \textbf{2.00} & 2.10 & 2.11 & \textbf{2.01} & 8.21 & \textbf{7.40} & 8.19 & \textbf{8.25} & \textbf{7.43} \
 Fingpt & 2.31 & \textbf{2.15} & 2.25 & 2.23 & \textbf{2.11} & 10.04 & \textbf{8.62} & 9.51 & \textbf{9.34} & \textbf{8.24} \
 Fiqa & 2.40 & \textbf{2.25} & 2.31 & 2.31 & \textbf{2.19} & 11.02 & \textbf{9.45} & 10.10 & \textbf{10.05} & \textbf{8.93} \
 Twitter & 2.21 & \textbf{2.10} & 2.21 & 2.20 & \textbf{2.09} & 9.14 & \textbf{8.18} & 9.10 & \textbf{8.99} & \textbf{8.05} \
 Wikitext & 2.24 & \textbf{2.11} & 2.21 & 2.19 & \textbf{2.08} & 9.41 & \textbf{8.23} & 9.08 & \textbf{8.89} & \textbf{8.00} \
\rowcolor{blue!10} \textbf{Average} & \textbf{2.27} & \textbf{2.13} & \textbf{2.21} & \textbf{2.20} & \textbf{2.09} & \textbf{9.69} & \textbf{8.42} & \textbf{9.18} & \textbf{9.02} & \textbf{8.09}  \
\hline
\end{tabular}
\end{table}



% Cross-dataset comparisons
% Cross-Dataset Comparison: Financial News as Evaluation Dataset
% Shows which training dataset performs best on Financial News
% Bold values indicate best performance for each model size

\begin{table}[h]
\centering
\caption{Financial News Evaluation: Performance Across Training Datasets}
\label{tab:cross_financial_news}
\resizebox{\textwidth}{!}{
\begin{tabular}{l|ccc|ccc}
\toprule
\multirow{2}{*}{\textbf{Training Dataset}} &
\multicolumn{3}{c|}{\textbf{Cross-Entropy Loss}} &
\multicolumn{3}{c}{\textbf{Perplexity}} \\
\cmidrule(lr){2-4} \cmidrule(lr){5-7}
& \textbf{0.6B} & \textbf{1.7B} & \textbf{4B} & \textbf{0.6B} & \textbf{1.7B} & \textbf{4B} \\
\midrule
Alpaca (2e-5) & 3.92 & 2.71 & 2.15 & 50.40 & 15.05 & 8.58 \\
Financial QA (2e-5) & \textbf{2.36} & \textbf{2.17} & 2.13 & \textbf{10.60} & \textbf{8.78} & 8.41 \\
Financial QA (1.7B: 1e-5, 4B: 5e-6) & \textbf{2.36} & 2.23 & 2.04 & \textbf{10.60} & 9.25 & 7.71 \\
FinGPT (2e-5) & 3.36 & 2.45 & 2.07 & 28.72 & 11.58 & 7.92 \\
FiQA (2e-5) & 3.90 & 2.54 & \textbf{2.01} & 49.22 & 12.74 & \textbf{7.43} \\
Mixed Financial (2e-5) & 4.03 & 3.05 & 2.63 & 56.35 & 21.19 & 13.84 \\
Mixed Wiki+Financial (2e-5) & 3.65 & 3.13 & 2.77 & 38.68 & 22.79 & 15.91 \\
Financial News (2e-5) & 3.96 & 3.13 & 2.86 & 52.25 & 22.91 & 17.47 \\
SEC Reports (2e-5) & 3.71 & 3.08 & 2.81 & 40.85 & 21.65 & 16.67 \\
Twitter Financial (2e-5) & 3.17 & 2.80 & 2.87 & 23.77 & 16.48 & 17.67 \\
Twitter Financial (1.7B: 1e-5, 4B: 5e-6) & 3.17 & 2.65 & 2.54 & 23.77 & 14.10 & 12.68 \\
WikiText (2e-5) & 2.62 & 2.93 & 3.37 & 13.70 & 18.78 & 29.19 \\
WikiText (1.7B: 5e-6, 4B: 3e-6) & 2.62 & 3.52 & 3.27 & 13.70 & 33.66 & 26.44 \\
\bottomrule
\end{tabular}
}
\end{table}


% Cross-Dataset Comparison: SEC Reports as Evaluation Dataset
% Shows which training dataset performs best on SEC Reports
% Bold values indicate best performance for each model size

\begin{table}[h]
\centering
\caption{SEC Reports Evaluation: Performance Across Training Datasets}
\label{tab:cross_financial_repor}
\begin{tabular}{l|ccc|ccc}
\hline
\textbf{Training Dataset} & \multicolumn{3}{c|}{\textbf{Cross-Entropy Loss}} & \multicolumn{3}{c}{\textbf{Perplexity}} \\n\cline{2-4} \cline{5-7}
  & \textbf{0.6B} & \textbf{1.7B} & \textbf{4B} & \textbf{0.6B} & \textbf{1.7B} & \textbf{4B} \\
Alpaca (2e-5) & 4.54 & 2.85 & 2.11 & 93.56 & 17.26 & 8.25  \
 Financial QA (2e-5) & 2.11 & \textbf{2.00} & 2.11 & 8.21 & \textbf{7.40} & 8.25  \
 Financial QA (1.7B: 1e-5, 4B: 5e-6) & 2.11 & 2.10 & 2.01 & 8.21 & 8.19 & 7.43  \
 FinGPT (2e-5) & 3.53 & 2.31 & 1.82 & 33.97 & 10.12 & 6.20  \
 FiQA (2e-5) & 4.42 & 2.53 & \textbf{1.81} & 83.48 & 12.51 & \textbf{6.14}  \
 Mixed Financial (2e-5) & 4.94 & 3.58 & 3.11 & 139.62 & 35.83 & 22.36  \
 Mixed Wiki+Financial (2e-5) & 4.35 & 3.69 & 3.33 & 77.57 & 40.17 & 27.91  \
 Financial News (2e-5) & 4.85 & 3.73 & 3.51 & 127.73 & 41.68 & 33.46  \
 SEC Reports (2e-5) & 3.72 & 2.96 & 2.77 & 41.12 & 19.36 & 15.91  \
 Twitter Financial (2e-5) & 2.48 & 2.32 & 2.80 & 11.95 & 10.17 & 16.42  \
 Twitter Financial (1.7B: 1e-5, 4B: 5e-6) & 2.48 & 2.16 & 2.39 & 11.95 & 8.70 & 10.93  \
 WikiText (2e-5) & \textbf{1.39} & 3.27 & 3.44 & \textbf{3.99} & 26.46 & 31.23  \
 WikiText (1.7B: 5e-6, 4B: 3e-6) & \textbf{1.39} & 3.91 & 3.75 & \textbf{3.99} & 49.83 & 42.41  \
\hline
\end{tabular}
\end{table}


% Cross-Dataset Comparison: Alpaca as Evaluation Dataset
% Shows which training dataset performs best on Alpaca
% Bold values indicate best performance for each model size

\begin{table}[h]
\centering
\caption[Alpaca Evaluation: Cross-Dataset Performance]{Alpaca Evaluation: Performance Across Training Datasets}
\label{tab:cross_alpaca}
\begin{tabular}{l|ccc|ccc}
\hline
\textbf{Training Dataset} & \multicolumn{3}{c|}{\textbf{Cross-Entropy Loss}} & \multicolumn{3}{c}{\textbf{Perplexity}} \\
\cline{2-4} \cline{5-7}
  & \textbf{0.6B} & \textbf{1.7B} & \textbf{4B} & \textbf{0.6B} & \textbf{1.7B} & \textbf{4B} \\
Alpaca (2e-5) & 4.16 & 2.75 & 2.11 & 63.73 & 15.61 & 8.22  \\
Financial QA (2e-5) & 2.38 & \textbf{2.23} & 2.29 & 10.82 & \textbf{9.31} & 9.91  \\
Financial QA (1.7B: 1e-5, 4B: 5e-6) & 2.38 & 2.29 & 2.18 & 10.82 & 9.92 & 8.88  \\
FinGPT (2e-5) & 3.57 & 2.55 & 2.11 & 35.55 & 12.78 & 8.27  \\
FiQA (2e-5) & 4.14 & 2.56 & \textbf{1.96} & 62.97 & 12.96 & \textbf{7.12}  \\
Mixed Financial (2e-5) & 4.54 & 3.38 & 2.97 & 93.35 & 29.53 & 19.50  \\
Mixed Wiki+Financial (2e-5) & 4.07 & 3.48 & 3.15 & 58.56 & 32.38 & 23.23  \\
Financial News (2e-5) & 4.57 & 3.61 & 3.39 & 96.31 & 36.92 & 29.75  \\
SEC Reports (2e-5) & 3.86 & 3.14 & 2.92 & 47.65 & 23.04 & 18.54  \\
Twitter Financial (2e-5) & 3.01 & 2.66 & 2.96 & 20.21 & 14.33 & 19.20  \\
Twitter Financial (1.7B: 1e-5, 4B: 5e-6) & 3.01 & 2.54 & 2.61 & 20.21 & 12.66 & 13.65  \\
WikiText (2e-5) & \textbf{2.22} & 3.24 & 3.48 & \textbf{9.23} & 25.51 & 32.38  \\
WikiText (1.7B: 5e-6, 4B: 3e-6) & \textbf{2.22} & 3.79 & 3.64 & \textbf{9.23} & 44.22 & 38.06  \\
\hline
\end{tabular}
\end{table}


% Cross-Dataset Comparison: FinGPT as Evaluation Dataset
% Shows which training dataset performs best on FinGPT
% Bold values indicate best performance for each model size

\begin{table}[h]
\centering
\caption{FinGPT Evaluation: Performance Across Training Datasets}
\label{tab:cross_fingpt}
\begin{tabular}{l|ccc|ccc}
\hline
\textbf{Training Dataset} & \multicolumn{3}{c|}{\textbf{Cross-Entropy Loss}} & \multicolumn{3}{c}{\textbf{Perplexity}} \\n\cline{2-4} \cline{5-7}
  & \textbf{0.6B} & \textbf{1.7B} & \textbf{4B} & \textbf{0.6B} & \textbf{1.7B} & \textbf{4B} \\
Alpaca (2e-5) & 4.71 & 2.99 & 2.22 & 111.65 & 19.85 & 9.18  \
 Financial QA (2e-5) & 2.31 & 2.15 & 2.23 & 10.04 & 8.62 & 9.34  \
 Financial QA (1.7B: 1e-5, 4B: 5e-6) & 2.31 & 2.25 & 2.11 & 10.04 & 9.51 & 8.24  \
 FinGPT (2e-5) & 3.49 & 2.26 & \textbf{1.74} & 32.78 & 9.56 & \textbf{5.67}  \
 FiQA (2e-5) & 4.67 & 2.71 & 1.95 & 107.25 & 15.08 & 7.01  \
 Mixed Financial (2e-5) & 5.04 & 3.63 & 3.14 & 153.94 & 37.82 & 23.08  \
 Mixed Wiki+Financial (2e-5) & 4.44 & 3.75 & 3.37 & 84.43 & 42.50 & 28.92  \
 Financial News (2e-5) & 5.08 & 3.90 & 3.64 & 160.92 & 49.56 & 38.03  \
 SEC Reports (2e-5) & 3.97 & 3.15 & 2.93 & 53.18 & 23.41 & 18.68  \
 Twitter Financial (2e-5) & 2.74 & 2.50 & 2.91 & 15.53 & 12.23 & 18.34  \
 Twitter Financial (1.7B: 1e-5, 4B: 5e-6) & 2.74 & 2.34 & 2.54 & 15.53 & 10.41 & 12.69  \
 WikiText (2e-5) & \textbf{1.30} & \textbf{2.11} & 3.57 & \textbf{3.67} & \textbf{8.27} & 35.50  \
 WikiText (1.7B: 5e-6, 4B: 3e-6) & \textbf{1.30} & 4.07 & 3.88 & \textbf{3.67} & 58.55 & 48.30  \
\hline
\end{tabular}
\end{table}


% Cross-Dataset Comparison: FiQA as Evaluation Dataset
% Shows which training dataset performs best on FiQA
% Bold values indicate best performance for each model size

\begin{table}[htbp]
\centering
\caption[FiQA Evaluation: Cross-Dataset Performance]{FiQA Evaluation: Performance Across Training Datasets}
\label{tab:cross_fiqa}
\begin{tabular}{l|ccc|ccc}
\hline
\textbf{Training Dataset} & \multicolumn{3}{c|}{\textbf{Cross-Entropy Loss}} & \multicolumn{3}{c}{\textbf{Perplexity}} \\
\cline{2-4} \cline{5-7}
  & \textbf{0.6B} & \textbf{1.7B} & \textbf{4B} & \textbf{0.6B} & \textbf{1.7B} & \textbf{4B} \\
Alpaca (2e-5) & 4.29 & 2.87 & 2.22 & 73.12 & 17.63 & 9.22  \\
Financial QA (2e-5) & 2.40 & \textbf{2.25} & 2.31 & 11.02 & \textbf{9.45} & 10.05  \\
Financial QA (1.7B: 1e-5, 4B: 5e-6) & 2.40 & 2.31 & 2.19 & 11.02 & 10.10 & 8.93  \\
FinGPT (2e-5) & 3.57 & 2.55 & 2.10 & 35.64 & 12.79 & 8.16  \\
FiQA (2e-5) & 4.17 & 2.56 & \textbf{1.96} & 64.75 & 12.99 & \textbf{7.08}  \\
Mixed Financial (2e-5) & 4.63 & 3.46 & 3.05 & 102.47 & 31.85 & 21.20  \\
Mixed Wiki+Financial (2e-5) & 4.14 & 3.56 & 3.24 & 63.03 & 35.04 & 25.61  \\
Financial News (2e-5) & 4.62 & 3.65 & 3.46 & 101.32 & 38.68 & 31.69  \\
SEC Reports (2e-5) & 3.85 & 3.14 & 2.96 & 47.22 & 23.15 & 19.34  \\
Twitter Financial (2e-5) & 2.98 & 2.66 & 3.00 & 19.67 & 14.26 & 20.09  \\
Twitter Financial (1.7B: 1e-5, 4B: 5e-6) & 2.98 & 2.50 & 2.61 & 19.67 & 12.20 & 13.61  \\
WikiText (2e-5) & \textbf{2.07} & 3.14 & 3.53 & \textbf{7.89} & 23.15 & 34.03  \\
WikiText (1.7B: 5e-6, 4B: 3e-6) & \textbf{2.07} & 3.85 & 3.74 & \textbf{7.89} & 46.81 & 42.04  \\
\hline
\end{tabular}
\end{table}


% Cross-Dataset Comparison: Twitter Financial as Evaluation Dataset
% Shows which training dataset performs best on Twitter Financial
% Bold values indicate best performance for each model size

\begin{table}[htbp]
\centering
\caption[Twitter Financial Evaluation: Cross-Dataset Performance]{Twitter Financial Evaluation: Performance Across Training Datasets}
\label{tab:cross_twitter}
\begin{tabular}{l|ccc|ccc}
\hline
\textbf{Training Dataset} & \multicolumn{3}{c|}{\textbf{Cross-Entropy Loss}} & \multicolumn{3}{c}{\textbf{Perplexity}} \\
\cline{2-4} \cline{5-7}
  & \textbf{0.6B} & \textbf{1.7B} & \textbf{4B} & \textbf{0.6B} & \textbf{1.7B} & \textbf{4B} \\
Alpaca (2e-5) & 4.78 & 2.99 & 2.19 & 118.74 & 19.82 & 8.97  \\
Financial QA (2e-5) & 2.21 & \textbf{2.10} & 2.20 & 9.14 & \textbf{8.18} & 8.99  \\
Financial QA (1.7B: 1e-5, 4B: 5e-6) & 2.21 & 2.21 & 2.09 & 9.14 & 9.10 & 8.05  \\
FinGPT (2e-5) & 3.68 & 2.40 & \textbf{1.87} & 39.54 & 11.05 & \textbf{6.46}  \\
FiQA (2e-5) & 4.66 & 2.65 & 1.88 & 105.32 & 14.10 & 6.58  \\
Mixed Financial (2e-5) & 5.21 & 3.76 & 3.25 & 182.63 & 42.91 & 25.72  \\
Mixed Wiki+Financial (2e-5) & 4.59 & 3.88 & 3.48 & 98.13 & 48.42 & 32.48  \\
Financial News (2e-5) & 5.11 & 3.91 & 3.66 & 165.22 & 49.88 & 38.98  \\
SEC Reports (2e-5) & 3.94 & 3.13 & 2.90 & 51.30 & 22.86 & 18.12  \\
Twitter Financial (2e-5) & 2.53 & 2.40 & 2.88 & 12.60 & 11.02 & 17.83  \\
Twitter Financial (1.7B: 1e-5, 4B: 5e-6) & 2.53 & 2.22 & 2.47 & 12.60 & 9.21 & 11.81  \\
WikiText (2e-5) & \textbf{1.45} & 2.78 & 3.52 & \textbf{4.26} & 16.06 & 33.71  \\
WikiText (1.7B: 5e-6, 4B: 3e-6) & \textbf{1.45} & 4.08 & 3.88 & \textbf{4.26} & 58.98 & 48.48  \\
\hline
\end{tabular}
\end{table}


% Cross-Dataset Comparison: Financial QA as Evaluation Dataset
% Shows which training dataset performs best on Financial QA
% Bold values indicate best performance for each model size

\begin{table}[h]
\centering
\caption{Financial QA Evaluation: Performance Across Training Datasets}
\label{tab:cross_financial_qa}
\begin{tabular}{l|ccc|ccc}
\hline
\textbf{Training Dataset} & \multicolumn{3}{c|}{\textbf{Cross-Entropy Loss}} & \multicolumn{3}{c}{\textbf{Perplexity}} \\n\cline{2-4} \cline{5-7}
  & \textbf{0.6B} & \textbf{1.7B} & \textbf{4B} & \textbf{0.6B} & \textbf{1.7B} & \textbf{4B} \\
Alpaca (2e-5) & 4.77 & 2.95 & 2.15 & 117.40 & 19.11 & 8.56  \
 Financial QA (2e-5) & \textbf{2.12} & \textbf{2.01} & 2.12 & \textbf{8.29} & \textbf{7.44} & 8.29  \
 Financial QA (1.7B: 1e-5, 4B: 5e-6) & \textbf{2.12} & 2.12 & 2.01 & \textbf{8.29} & 8.29 & 7.43  \
 FinGPT (2e-5) & 3.66 & 2.38 & \textbf{1.83} & 38.96 & 10.85 & \textbf{6.24}  \
 FiQA (2e-5) & 4.64 & 2.60 & 1.84 & 103.40 & 13.53 & 6.32  \
 Mixed Financial (2e-5) & 5.21 & 3.75 & 3.23 & 183.72 & 42.30 & 25.14  \
 Mixed Wiki+Financial (2e-5) & 4.58 & 3.87 & 3.46 & 97.49 & 47.94 & 31.76  \
 Financial News (2e-5) & 5.11 & 3.90 & 3.66 & 166.10 & 49.53 & 38.90  \
 SEC Reports (2e-5) & 3.90 & 3.08 & 2.86 & 49.30 & 21.77 & 17.39  \
 Twitter Financial (2e-5) & 2.46 & 2.32 & 2.83 & 11.76 & 10.15 & 16.98  \
 Twitter Financial (1.7B: 1e-5, 4B: 5e-6) & 2.46 & 2.16 & 2.43 & 11.76 & 8.69 & 11.39  \
 WikiText (2e-5) & 3.40 & 10.67 & 3.37 & 29.90 & $\infty$ & 29.08  \
 WikiText (1.7B: 5e-6, 4B: 3e-6) & 3.40 & 4.07 & 3.87 & 29.90 & 58.33 & 47.98  \
\hline
\end{tabular}
\end{table}


% Cross-Dataset Comparison: WikiText as Evaluation Dataset
% Shows which training dataset performs best on WikiText
% Bold values indicate best performance for each model size

\begin{table}[htbp]
\centering
\caption[WikiText Evaluation: Cross-Dataset Performance]{WikiText Evaluation: Performance Across Training Datasets}
\label{tab:cross_wikitext}
\begin{tabular}{l|ccc|ccc}
\hline
\textbf{Training Dataset} & \multicolumn{3}{c|}{\textbf{Cross-Entropy Loss}} & \multicolumn{3}{c}{\textbf{Perplexity}} \\
\cline{2-4} \cline{5-7}
  & \textbf{0.6B} & \textbf{1.7B} & \textbf{4B} & \textbf{0.6B} & \textbf{1.7B} & \textbf{4B} \\
Alpaca (2e-5) & 4.63 & 2.94 & 2.18 & 102.41 & 18.85 & 8.88  \\
Financial QA (2e-5) & 2.24 & \textbf{2.11} & 2.19 & 9.41 & \textbf{8.23} & 8.89  \\
Financial QA (1.7B: 1e-5, 4B: 5e-6) & 2.24 & 2.21 & 2.08 & 9.41 & 9.08 & 8.00  \\
FinGPT (2e-5) & 3.66 & 2.44 & 1.99 & 38.70 & 11.46 & 7.29  \\
FiQA (2e-5) & 4.52 & 2.63 & \textbf{1.91} & 92.13 & 13.81 & \textbf{6.72}  \\
Mixed Wiki+Financial (2e-5) & 4.41 & 3.74 & 3.32 & 82.10 & 41.95 & 27.72  \\
Financial News (2e-5) & 4.95 & 3.81 & 3.54 & 140.71 & 45.17 & 34.33  \\
SEC Reports (2e-5) & 3.89 & 3.10 & 2.88 & 49.02 & 22.21 & 17.72  \\
Twitter Financial (2e-5) & 2.69 & 2.47 & 2.88 & 14.74 & 11.78 & 17.85  \\
Twitter Financial (1.7B: 1e-5, 4B: 5e-6) & 2.69 & 2.30 & 2.49 & 14.74 & 9.94 & 12.02  \\
WikiText (2e-5) & \textbf{1.56} & 3.42 & 3.30 & \textbf{4.78} & 30.63 & 27.19  \\
WikiText (1.7B: 5e-6, 4B: 3e-6) & \textbf{1.56} & 3.88 & 3.65 & \textbf{4.78} & 48.44 & 38.60  \\
\hline
\end{tabular}
\end{table}



\chapter{Discussion}

This chapter interprets the experimental findings from Chapter 4, explaining mechanisms behind data mixture effects, training dynamics, and generalization patterns. We synthesize empirical observations into actionable guidelines and acknowledge methodological limitations. Still, we keep conclusions tied to our evidence. Where the data are thin, we say so.

\section{Key Empirical Findings}

Our 10 experiments (30 models, 237 evaluations) lead to four main findings about data mixture effects in specialized-domain language model pretraining:

\textbf{Finding 1: In-Domain Diversity Outperforms General Corpus Quality}

Mixed Financial datasets achieved 21.55 ppl (4B) with 55\% relative spread, substantially better than WikiText's 41.96 ppl mean financial performance (\~53\% relative spread, after LR adjustment). This 1.95$\times$ performance gap demonstrates that multiple in-domain datasets—even if individually small (Twitter 0.3M tokens) or noisy (social media text)—provide superior domain specialization compared to large, curated general corpora. The result challenges conventional wisdom that high-quality general pretraining suffices for domain adaptation. \Cref{fig:scaling_comparison_all} visually confirms this hierarchy: the performance gap between Mixed Financial (blue line) and WikiText (green line) widens from 0.6B to 4B, indicating that in-domain diversity scales better than general quality. The cross-dataset tables (\Cref{tab:cross_financial_news,tab:cross_financial_qa,tab:cross_fingpt,tab:cross_fiqa}) further validate this through boldface patterns—Mixed Financial rows consistently capture best-performance positions across evaluation datasets, while WikiText rows rarely achieve boldface except in their own domain.

\textbf{Finding 2: Simple LR Reductions Stabilized a Few Runs}

All main runs used LR=2e-5. In three configurations that showed abnormalities (WikiText, Financial QA, Twitter), we retried with smaller learning rates (e.g., $1\times10^{-5}$ or $5\times10^{-6}$) and observed improved stability and performance. We treat these reductions as pragmatic fixes in our setup rather than a general scaling rule. The differences between solid (original LR) and dashed (reduced LR) lines in \Cref{fig:scaling_wikitext,fig:scaling_financial_qa,fig:scaling_twitter} illustrate these context-specific improvements; \Cref{tab:financial_qa_lr_comparison,tab:twitter_lr_comparison} quantify them.

\textbf{Finding 3: Dataset Size Critically Affects Pretraining Viability}

Clear thresholds emerged: datasets $>$ 100M tokens support standalone pretraining (2–5 epochs, consistent generalization); 20–100M tokens are viable with caveats (6–30 epochs, moderate generalization); $<$ 20M tokens are not viable alone (67–249 epochs, severe overtraining and high cross-dataset variability). Correlation between log(tokens) and variability: $r = -0.78$ ($p < 0.01$). Small datasets require mixing regardless of optimization quality—data scarcity, not hyperparameters, limits performance. The scaling figures show this: \Cref{fig:scaling_news_articles,fig:scaling_sec_reports} (large datasets) have smooth curves with small gaps between sizes; \Cref{fig:scaling_financial_qa,fig:scaling_twitter} (small datasets) are erratic and need LR interventions. \Cref{tab:cross_financial_qa,tab:cross_twitter} reveal the brittleness: these rows achieve boldface mainly in their own columns (specialization) while showing 30–50 ppl elsewhere (transfer failure).

\textbf{Finding 4: Format Drives Transfer More Than Domain Vocabulary}

Document format and task structure predict cross-dataset transfer better than topical domain. Long-form documents (News $\leftrightarrow$ SEC: $r = 0.82$) transfer well despite style differences; instruction tasks cluster (FinGPT/Alpaca/FiQA: $r = 0.68-0.73$); short-form Twitter is isolated with high variability. A News model transfers better to regulatory SEC filings (both long-form, different domains) than to Twitter finance posts (same domain, different format). This suggests pretraining corpora should prioritize format diversity alongside domain coverage. The cross-dataset tables provide striking visual evidence: \Cref{tab:cross_financial_news,tab:cross_financial_repor} show boldface clustering along the News-SEC diagonal, confirming bidirectional long-form transfer. \Cref{tab:cross_alpaca,tab:cross_fingpt,tab:cross_fiqa} exhibit similar diagonal boldface patterns plus adjacency (instruction-trained models capturing boldface in each other's columns), demonstrating format-based clustering. In contrast, \Cref{tab:cross_twitter} shows complete isolation—boldface appears only in Twitter's own column regardless of which training dataset is used, visualizing the distributional uniqueness of short-form social media text.

These findings generalize beyond finance to any specialized-domain pretraining scenario where practitioners face similar trade-offs: domain vs general data, mixture composition, model scaling, and format diversity.

\section{Interpretation of Data Interaction Effects}

\subsection{Why WikiText Underperforms on Financial Tasks}

WikiText's catastrophic financial transfer (41.96 ppl mean vs 21.55 ppl for Mixed Financial) stems from three fundamental mismatches:

\textbf{1. Vocabulary Gap}: Financial language contains specialized terminology absent in encyclopedic text. Terms like ``EBITDA'' (earnings before interest, taxes, depreciation, amortization), ``alpha'' (excess returns), ``basis points'' (0.01\%), ``volatility'' (price fluctuation measure), ``hedging'' (risk mitigation strategy), and ``P/E ratio'' (price-to-earnings valuation) rarely appear in Wikipedia. When WikiText models encounter financial evaluation texts, they face effective out-of-vocabulary scenarios despite shared syntactic structure. The model's vocabulary distribution mismatches the evaluation domain's lexical requirements.

\textbf{2. Reasoning Pattern Mismatch}: Financial analysis requires forward-looking causal reasoning: ``Company X's earnings miss will pressure the stock downward'' (cause-effect prediction), ``Rising interest rates typically compress equity valuations'' (conditional reasoning), ``The Fed's hawkish stance suggests tightening ahead'' (implicit reasoning from policy to outcomes). Wikipedia's encyclopedic, descriptive style—focused on established facts, historical narratives, and definitional content—doesn't exercise these prospective reasoning patterns. Models pretrained on WikiText learn to predict continuations based on factual descriptions, not anticipatory financial logic.

\textbf{3. Discourse Structure Divergence}: Financial news follows inverted pyramid structure (conclusion first, then supporting details); earnings reports have standardized sections (forward-looking statements, risk factors, MD\&A); analyst reports use comparison tables and numerical evidence. Wikipedia articles employ chronological narratives (biographical entries), topical organization (scientific articles), or definitional structures (concept entries). These discourse patterns create different coherence signals—WikiText models learn topic progression and factual elaboration, while financial texts require comparative analysis and evidential reasoning structures.

\textbf{Why General → Financial Transfer Fails But Financial → General Succeeds}: The asymmetry (WikiText @ 4B: 41.96 ppl financial vs Mixed Financial @ 4B: 27.72 ppl WikiText) reveals hierarchical structure. General language (syntax, semantics, discourse coherence) forms a foundation; financial language adds specialized vocabulary and reasoning on top. Starting from general pretraining provides linguistic prerequisites; domain-specific training adds specialization without catastrophic forgetting of fundamentals. Conversely, starting from general pretraining lacks domain prerequisites—vocabulary and reasoning gaps cannot be bridged by linguistic competence alone. This asymmetry is strikingly visible in \Cref{tab:cross_wikitext}: WikiText training rows show boldface in WikiText columns (4.78-38.60 ppl across model sizes) but poor financial performance (26-58 ppl depending on dataset and LR). Financial training rows show acceptable WikiText performance (27-42 ppl) alongside superior financial metrics. The table's boldface distribution pattern—concentrated in financial rows for most columns, scattered in WikiText rows—quantitatively demonstrates that financial pretraining retains general capability while general pretraining fails to acquire domain specialization.

\subsection{Benefits of In-Domain Diversity}

Mixed Financial's advantage (21.55 ppl, 55\% relative spread) over individual datasets (mean: 24.8 ppl, \~65\% relative spread) and WikiText (41.96 ppl financial, \~53\% relative spread after LR adjustment) stems from diversity-driven stability:

\textbf{Cross-Format Exposure}: The 7-dataset mixture spans long-form documents (News 197M, SEC 80M), instruction formats (FinGPT 19M, Alpaca 17M, FiQA 4M), and short-form text (Twitter 0.3M, Financial QA 3.5M). This format diversity prevents overfitting to structural artifacts. Models trained on pure News learn long-form coherence but fail on dialogic Q\&A (41\% worse on FiQA); mixed models handle both, averaging only 30\% degradation across all formats.

\textbf{Vocabulary Coverage}: Different financial datasets emphasize different lexical subdomains: News covers market events and company names; SEC covers regulatory terminology (``10-K'', ``forward-looking statements''); FinGPT covers sentiment vocabulary (``bullish'', ``bearish''); Alpaca covers financial concepts (``compound interest'', ``diversification''). The mixture creates broad vocabulary coverage—no single dataset provides this breadth. Mixed models encounter 3.2$\times$ more unique financial terms during training than the largest individual dataset (News), improving lexical stability.

\textbf{Task Diversity Regularization}: Mixing datasets with different objectives (sentiment classification, Q\&A, document completion) acts as implicit multi-task learning. The model cannot overfit to any single task's superficial cues (e.g., specific sentiment indicators in FinGPT, formulaic question structures in Alpaca) because the loss function averages across diverse distributions. This produces representations that capture underlying financial semantics rather than task-specific shortcuts.

\textbf{Preventing Data Memorization}: Small datasets suffer from memorization—Financial QA (3.5M tokens, 67-100 epochs) achieves 8.09 ppl in-domain but 41.7 ppl cross-dataset. The model memorizes training examples rather than learning generalizable patterns. Mixing prevents memorization by capping each dataset's contribution (50cap strategy limits News to 50\%, ensuring others get exposure) and diversifying the training distribution. Mixed models see fewer repeated examples from any single source, forcing extraction of transferable features.

\textbf{Quantitative Evidence}: Variability reduction correlates with mixture diversity: the 7-dataset mixture (\~55\% relative spread) compares favorably to individual datasets (often \~65\% or higher). The mixture improves both performance (21.55 vs 24.8 ppl mean) and consistency simultaneously. The cross-dataset tables illustrate this: Mixed Financial rows appear most frequently in boldface across evaluation columns. Individual dataset rows (News, SEC, FinGPT, etc.) capture boldface mainly in their own or nearby columns, while Mixed Financial remains competitive across the board. This broad vs narrow boldface distribution visualizes how diversity enables more stable generalization across heterogeneous evaluation scenarios.

\subsection{Domain Interference Patterns}

While in-domain diversity helps, cross-domain mixing (Mixed Wiki+Financial) shows interference:

\textbf{Performance-Diversity Trade-off}: Mixed Wiki+Financial achieves 26.69 ppl (4B), 24\% worse than pure Mixed Financial (21.55 ppl), despite including WikiText. On WikiText specifically, the mixed approach improves performance modestly compared to pure Financial, but mean financial performance degrades notably. The trade-off is unfavorable for finance-focused applications: sacrificing financial performance for a small general-domain gain.

\textbf{Competing Optimization Signals}: Financial and general domains create conflicting gradients. Financial texts reward predicting domain terminology (``EBITDA'' following ``reported''); general texts reward different continuations (``findings'' following ``reported''). The model's parameters cannot simultaneously optimize for both distributions without compromise. Mixed Wiki+Financial models average these signals, achieving moderate performance on both rather than excellence on either. The 62\% variance (vs 55\% pure financial) reflects this optimization conflict.

\textbf{When Mixing Hurts vs Helps}: Intra-domain mixing helps because datasets share core semantics (financial vocabulary, reasoning patterns) while differing in format and task type—diversity reinforces fundamentals. Cross-domain mixing hurts when domains diverge in vocabulary and reasoning (encyclopedic vs analytical), creating zero-sum trade-offs. The 50cap strategy mitigates but doesn't eliminate interference: capping WikiText at 50\% limits damage but still dilutes financial specialization. This distinction is evident comparing \Cref{tab:mixed_financial_results} (pure financial mixture) and \Cref{tab:mixed_wiki_financial_results} (cross-domain mixture): the former shows consistently lower perplexity across all financial evaluation datasets, with the performance advantage increasing at larger model sizes. \Cref{fig:scaling_mixed_financial,fig:scaling_mixed_wiki_financial} visually confirm this—the pure financial mixture (first figure) shows steeper slope (22.6\% total improvement) compared to Wiki+Financial (second figure, 15.1\% improvement), indicating that domain conflict reduces scaling efficiency.

\textbf{Practical Implication}: For specialized applications, domain purity wins. Only mix cross-domain when explicit general-domain retention is required (e.g., conversational agents handling both financial and general queries). For finance-focused deployments, pure in-domain mixtures maximize performance.

\subsection{Scale-Dependent Training Notes}

Our experience suggests that larger models can be more sensitive to optimization settings on some datasets. While we kept LR=2e-5 for main runs, reducing LR in a handful of follow-ups helped stabilize training. We do not claim a general rule beyond this observation.

\section{Practical Guidelines for Financial LM Pretraining}

Synthesizing experimental findings into actionable recommendations:

\subsection{Data Mixture Strategies by Use Case}

\textbf{General-Purpose Financial NLP}: Use Mixed Financial (7 datasets, 50cap). Achieves best all-around performance (21.55 ppl, 55\% relative spread) with stable cross-task generalization. Suitable for applications requiring diverse financial capabilities: sentiment analysis, document summarization, Q\&A, information extraction. As shown in \Cref{fig:scaling_mixed_financial,fig:scaling_comparison_all}, this approach scales reliably across model sizes and consistently outperforms alternatives. The cross-dataset tables also support this choice: Mixed Financial rows capture boldface positions more often than any individual dataset across the eight evaluation scenarios.

\textbf{Specialized Document Analysis}: Use single large dataset if available ($>$ 100M tokens). SEC @ 4B (15.91 ppl on SEC; \~19\% relative spread across evaluations) excels for regulatory filing analysis; News @ 4B (17.47 ppl on News; \~66\% relative spread) excels for journalism. Specialization improves in-domain performance but sacrifices cross-format transfer. \Cref{fig:scaling_news_articles,fig:scaling_sec_reports} show these datasets maintain stable scaling without requiring LR adjustments. However, \Cref{tab:cross_financial_news,tab:cross_financial_repor} reveal that News and SEC training rows achieve boldface primarily within document-format columns, confirming limited format diversity.

\textbf{Instruction-Following / Q\&A Applications}: Use FiQA (4M tokens, 16.35 ppl) or FinGPT (19M tokens, 19.83 ppl) for specialized Q\&A, or include in mixture for general applications. Instruction formats transfer moderately within task type ($r = 0.68-0.73$) but poorly to documents. The instruction-following tables (\Cref{tab:cross_alpaca,tab:cross_fingpt,tab:cross_fiqa}) show boldface clustering along the diagonal and adjacent instruction rows, visualizing the format-based transfer limitation.

\textbf{Balanced General + Financial Capabilities}: Use Mixed Wiki+Financial only if general-domain retention is explicitly required (e.g., chatbots handling both financial and general queries). Accepts 24\% financial performance cost for 16\% general improvement—unfavorable for finance-focused deployments. \Cref{fig:scaling_mixed_wiki_financial} shows reduced slope compared to pure financial mixture, and \Cref{tab:mixed_wiki_financial_results} documents the performance cost across all financial evaluation datasets.

\textbf{Avoid}: Pure WikiText for financial applications (2.0$\times$ performance degradation vs Mixed Financial on average across financial tasks), small individual datasets $<$ 20M tokens (non-viable standalone due to severe overtraining and high variability), single-format training when diverse tasks are expected (format mismatch prevents transfer). \Cref{fig:scaling_wikitext,fig:scaling_financial_qa,fig:scaling_twitter} provide visual evidence: WikiText requires LR adjustment and still shows poor financial transfer, while small datasets exhibit brittleness visible in both scaling curves and cross-dataset table patterns.

\subsection{Model Size Selection}

\textbf{0.6B Models}: Fast training ($\sim$6 hours for 100M tokens on Lambda Labs GPUs), low memory (4GB), suitable for rapid prototyping. Performance is acceptable for exploratory work, but variability is high (Mixed Financial: \~98\% relative spread). Use for development, experimentation, or extremely resource-constrained deployment (mobile devices).

\textbf{1.7B Models}: Best performance-efficiency balance. Training moderate ($\sim$12 hours), memory reasonable (10GB), performance strong with improved consistency vs 0.6B (Mixed Financial: \~63\% relative spread). Recommended for most applications—strong performance at substantially lower resource cost than 4B. Optimal for production deployment balancing quality and resource constraints.

\textbf{4B Models}: Best absolute performance (21.55 ppl, 55\% relative spread) but requires careful hyperparameter tuning (LR $5 \times 10^{-6}$ in affected cases) and substantial resources (20GB memory, $\sim$24 hours training). Use when maximizing performance justifies cost, and when expertise for hyperparameter tuning is available. Critical: failure to tune learning rate can cause reverse scaling—practitioners may need to reduce LR substantially at larger scales.

\textbf{Scaling Decision Tree}:
\begin{enumerate}
\item \textbf{Resource-constrained} (mobile, edge devices): 0.6B, accept 22\% performance loss vs 4B
\item \textbf{Balanced production deployment}: 1.7B, optimal trade-off (92\% of 4B performance, 50\% resources)
\item \textbf{Performance-critical} (willing to invest tuning effort): 4B, requires LR scaling expertise
\end{enumerate}

\subsection{Learning Rate Notes}

\textbf{Main setting}: $2 \times 10^{-5}$ across all primary experiments.

\textbf{Follow-ups}: For the few runs with anomalies, we used smaller LRs (e.g., $1\times10^{-5}$ or $5\times10^{-6}$) to stabilize training.

\textbf{Scope}: These are practical notes from our setup, not prescriptive guidelines.

\subsection{Token Budget Allocation}

\textbf{Optimal Token Budget}: 100M tokens sufficient when properly mixed across diverse datasets. Diminishing returns beyond this threshold for 0.6B-4B models in our experiments. Larger models ($>$ 7B) may benefit from extended training (200-500M tokens), but this remains untested.

\textbf{Mixture Composition}: Use 50cap strategy to prevent dominance. For $n$ datasets with sizes $\{s_1, s_2, ..., s_n\}$ where $s_1 > 0.5 \sum_i s_i$: cap $s_1$ at 50\% of total, sample others proportionally. This ensures diversity while respecting relative dataset informativeness.

\textbf{Sampling Strategy}: Token-level interleaving, not batch-level or epoch-level. Sample each training batch from mixture distribution with probabilities proportional to (capped) dataset sizes. Avoids sequential exposure that can cause catastrophic forgetting.

\textbf{Dataset Prioritization}: When curating datasets, prioritize: (1) Format diversity (documents, Q\&A, dialogue), (2) Size (aim for $\geq$ 100M total across sources), (3) Quality (clean text $>$ noisy text, but in-domain noisy $>$ out-of-domain clean). Don't exclude small datasets ($<$ 20M tokens) from mixtures—they contribute valuable diversity despite non-viability standalone.

\section{Limitations and Threats to Validity}

\textbf{Single Model Family}: All experiments used Qwen3 (0.6B/1.7B/4B). Observations about LR behavior may be architecture- and dataset-specific. Other decoder-only transformers (LLaMA, Gemma, Phi) could behave differently; validation required. Encoder-only (BERT) or encoder-decoder (T5) models may show different mixture effects due to bidirectional attention or different pretraining objectives.

\textbf{Fixed Mixture Strategy}: We used 50cap exclusively. Other algorithms (temperature sampling, equal mixing, DoReMi dynamic weighting) remain unexplored. The 50cap heuristic worked well but may not be optimal—ablation studies varying cap thresholds (30\%, 40\%, 60\%) could reveal improvements. Dynamic mixture strategies that adjust dataset weights during training based on validation loss may outperform static 50cap.

\textbf{Evaluation on Pretraining Distributions}: We evaluated using perplexity on held-out test sets from the same distributions as training data. This measures pretraining quality but doesn't directly assess downstream task performance. Fine-tuned performance on financial NLP tasks (sentiment classification accuracy, Q\&A F1, summarization ROUGE) may differ from pretraining perplexity rankings. Future work should validate that Mixed Financial's pretraining advantage transfers to downstream applications.

\textbf{Hardware Constraints}: Experiments limited to 0.6B-4B models due to available hardware (RTX A6000 48GB, A100 40GB, H100 80GB rented from Lambda Labs). Larger models (7B, 13B, 70B) may show different patterns; mixture benefits may increase or decrease with scale. We did not investigate LR behavior beyond the few follow-ups reported here.

\textbf{Limited Hyperparameter Search}: We systematically explored learning rates but kept other hyperparameters fixed (effective batch size 8, warmup 1000 steps, cosine schedule). Larger hyperparameter sweeps over batch size (4, 8, 16, 32), warmup ratios (1\%, 3\%, 5\%), and schedules (linear, cosine, polynomial) may reveal better configurations. Computational budget constraints prevented exhaustive search.

\textbf{Financial Domain Specificity}: Results may not generalize to other specialized domains with different characteristics. Legal text (extremely long documents, formal citations) or medical text (heavy abbreviations, multimodal integration) may show different mixture effects. The core principles (in-domain diversity, and our LR heuristics) may generalize, but specific mixture ratios and optimal configurations require domain-specific validation.

Despite these limitations, our findings provide solid empirical evidence for data mixture effects, training dynamics, and practical practices applicable to financial LM pretraining and likely informative for other specialized domains.

\chapter{Conclusion}


This thesis shows that effective specialized language models can be developed without massive computational resources or diverse data mixtures. By selecting focused medium datasets (3.6–8.5M tokens), using stable training settings, and targeting lightweight 0.6–4B parameter models, one can train privacy-preserving financial NLP systems suitable for on-device deployment. Contrary to expectations, individual datasets (FiQA, FinGPT, Alpaca) consistently outperform mixtures on both performance (2.5–3.2$\times$ better) and consistency (1.5–4.8$\times$ better).

The core insight that {individual medium datasets (3.6–8.5M tokens) consistently outperform mixtures on both performance and consistency} challenges conventional belief favoring data diversity. At fixed token budgets (100M), FiQA/FinGPT/Alpaca achieve 2.5–3.2$\times$ better perplexity and 1.5–4.8$\times$ better consistency than 7-dataset mixtures. Format inconsistency, differences in vocabulary distribution, and multi-task interference degrade mixture performance despite anticipated diversity benefits. From our experiments and findings, we argue that specialized pretraining should prioritize focused, high-quality medium datasets over diverse mixtures, especially when token budgets are limited. For domains with curated data (finance, legal, medical), individual dataset optimization offers superior performance at lower cost than either mixtures or general-purpose model adaptation.

As privacy regulations tighten and organizations recognize competitive value in proprietary data, on-device specialized models will become increasingly important. This work provides empirical foundations and practical guidelines for developing such systems where powerful NLP capabilities are accessible while at the same time also ensuring privacy and low cost requirements.


%----------------------------------------------------------------------------------------
% BIBLIOGRAPHY
%----------------------------------------------------------------------------------------
\printbibliography

\newpage

%----------------------------------------------------------------------------------------
% DECLARATION
%----------------------------------------------------------------------------------------
\thispagestyle{firststyle}

\section*{Eidesstattliche Erklärung}
Der/Die Verfasser/in erklärt an Eides statt, dass er/sie die vorliegende Arbeit selbständig, ohne fremde Hilfe und ohne Benutzung anderer als die angegebenen Hilfsmittel angefertigt hat. Die aus fremden Quellen (einschliesslich elektronischer Quellen) direkt oder indirekt übernommenen Gedanken sind ausnahmslos als solche kenntlich gemacht. Die Arbeit ist in gleicher oder ähnlicher Form oder auszugsweise im Rahmen einer anderen Prüfung noch nicht vorgelegt worden.\\[2cm]
\dotbox{Ort, Datum} \hfill \dotbox{Unterschrift des/der Verfassers/in}

\end{document}
